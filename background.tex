\section{Background}

We start by giving a basic introduction to our source and target
languages, focusing on the parts relevant to our work. We will discuss finer
semantic details where needed in \autoref{sec:trans} and \autoref{sec:mutref}.

\subsection{Rust}
\label{sec:rust}

Rust~\cite{matsakis2014rust} is a modern, multi-paradigm systems programming language sponsored
by Mozilla Research and developed as an open-source community effort. Rust is still a quite young language, with its first stable
version having been released on May 15, 2015. The two biggest Rust projects as of
this writing are the Servo\footnote{\url{https://github.com/servo/servo}}~\cite{anderson2016engineering} web browser engine and
the Rust compiler \texttt{rustc}\footnote{\url{https://github.com/rust-lang/rust}} itself.

As a partly functional language, Rust is primarily inspired by ML and shares much of
its syntax, as evidenced in \autoref{lst:rustml}. However, the syntax also shows
influences by C, the dominant systems programming language at present.
Finally, Rust also features a \emph{trait} system modeled after Haskell's type classes.

\begin{listing}[!bp]
  \inputminted{rust}{code/rustml.rs}
  
  \caption{A first example of functional programming in Rust, showing algebraic
    data types, polymorphic and higher-order functions, pattern matching, type
    inference and the expression-oriented syntax}
  \label{lst:rustml}
\end{listing}

Many features of Rust other than the syntax can be explained by Rust's desire to
feature an ML-like abstraction level while still running as efficiently as C,
even on resource-constrained systems that may not allow dynamic allocation at all.
Most prominently, Rust uses manual memory management just like C and C++, but
guarantees memory safety through its \emph{ownership} and
\emph{borrowing} systems. Rust also features an \emph{unsafe} language subset that allows
everything-goes programming on the level of C, but which is usually reserved for
implementing low-level primitives on which the \emph{safe} part of the language can
then build. In general, safe Rust is (thought to be) a type-safe
language like ML and unlike either C or C++. We focus on safe Rust in the
following and in our work in order to peruse these guarantees.

Ownership describes the application of \emph{linear types} to memory management
as proposed by Wadler~\cite{wadler1990linear}. The owner of a Rust object is the binding that is responsible for freeing the
object's resources (by calling a method of the \texttt{Drop} trait), which
generally happens at the end of the binding's scope.
Because an object managing resources should only ever have one owner, types that implement
\texttt{Drop} are linear types: A value may only be used once, at which point it is
consumed and ownership is transferred to its new binding.\footnote{Technically,
  because leaking resources (\ie not consuming the object at all) is a safe operation in Rust, such types are merely
  \emph{affine}. However, the distinction is not relevant for our purposes.
} In the following example, we extract an element from a \texttt{Vec} (a dynamically-sized array
type that has to free heap space in its \texttt{Drop} implementation), after which we are not permitted to use the \texttt{Vec} again.

\begin{minted}{rust}
fn get<T>(v: Vec<T>, idx: usize) -> T {
    v[idx]
    // v will be freed here
}

let v: Vec<u32> = vec![1];
let x = get(v, 0);
// get(v, 1); // error[E0382]: use of moved value: `v`
\end{minted}

One way to retain access to the \texttt{Vec} would be to also return it from the
function, regaining ownership. However, since \texttt{T} in general is a linear
type too, \texttt{get} would have to remove the indexed element before returning
the \texttt{Vec}.

A much better alternative is to use \emph{references}, which provide standard
pointer indirection. Because a reference does not take ownership of the pointee,
creating it is also called \emph{borrowing}.

\begin{minted}{rust}
fn get<T>(v: &Vec<T>, idx: usize) -> &T {
    &v[idx]
}

let v: Vec<u32> = vec![1];
let x = get(&v, 0); // x: &u32
\end{minted}

Here \rust{&T} represents an immutable reference to a value of type \rust{T}. Note that the compiler would stop us if we tried to return \texttt{v[idx]} by value:

\begin{verbatim}
error[E0507]: cannot move out of indexed content
\end{verbatim}

Still, coming from other languages with manual memory management, this might
look like a potentially unsafe thing to do: The function signature does not tell
the callee that the returned reference is only valid as long as the
\texttt{Vec}. Even Wadler tells us that a temporary reference to a linear value
must be checked not to escape from the local scope. Indeed, it seems like the following program should produce a dangling pointer.

\begin{minted}{rust}
fn dangling() -> u32 {
    let x = {
        let v: Vec<u32> = vec![1];
        get(&v, 0)
        // v will be freed here
    };
    *x
}
\end{minted}

However, the Rust compiler will stop us from doing this, printing an
elaborate error message:

\begin{minted}{text}
error: `v` does not live long enough
|
|         get(&v, 0)
|              ^ does not live long enough
|     };
|     - borrowed value only lives until here
|     *x
| }
| - borrowed value needs to live until here
\end{minted}

The compiler must have had some information about the relationship of \texttt{x}
and \texttt{v} in order to deduce this without resorting to inter-procedural
analysis. It turns out that the full signature of the \texttt{get} function is as follows:

\begin{minted}{rust}
fn get<'a, T>(v: &'a Vec<T>, idx: usize) -> &'a T
\end{minted}

\rust!'a! is called a \emph{formal lifetime parameter}. It
specifies that the returned reference is indeed only valid as long as the first
argument. By integrating lifetimes into the type system like this, Rust can
reason about references even when confronted with complex, inter-procedural, higher-order reference lifetime relations.

While we have solved the dangling pointer problem for immutable data, mutability
as so often aggravates the problem.

\begin{minted}{rust}
fn dangling2() -> u32 {
    let mut v: Vec<u32> = vec![1];
    let x = get(&v, 0);
    // remove all elements from v
    v.clear(); // shorthand for (&mut v).clear();
    *x
    // v will be freed here
}
\end{minted}

By clearing the vector while we still hold a reference to its content, we should
again produce a dangling pointer -- even though this time, \rust{v} indeed
outlives \rust{x}. Fortunately, the Rust compiler will again stop us:

\begin{minted}{text}
error[E0502]: cannot borrow `v` as mutable because it is also borrowed as immutable
|
|     let x = get(&v, 0);
|                  - immutable borrow occurs here
|     // remove all elements from v
|     v.clear(); // shorthand for (&mut v).clear();
|     ^ mutable borrow occurs here
|     *x
| }
| - immutable borrow ends here
\end{minted}

We have finally arrived at the aliasing problem: In a language with manual
memory management, we can create type unsafety through the mere existence of two
pointers, at least one of them mutable, to the same datum. Thus, Rust detects
and forbids any occurrences of mutable aliasing, as shown above.

%The beauty of Rust's solution to safe managed memory management is that the absence of mutable aliasing solves
The beauty of forbidding mutable aliasing is that it solves many sources of bugs
in imperative programs even outside of managed memory management. Indeed, as
Wadler notes, it makes mutable references safe even in a referentially
transparent language: ``In order for destructive updating of a value to be safe,
it is essential that there be only one reference to the value when the update
occurs''~\cite{wadler1990linear}. While Rust does introduce APIs, such as for I/O, that break referential
transparency, the absence of mutable aliasing still provides safety guarantees
that are usually only attributed to purely functional languages, first and
foremost among them the elimination of data races. By focusing on a subset of
Rust and its APIs that is truly referentially transparent, we obtain a
sufficiently narrow gap between Rust and the purely functional language Lean
that our transformation between them becomes feasible.

\subsection{Lean}

The Lean~\cite{de2015lean} theorem prover is an open source, dependently typed,
interactive theorem prover developed jointly at Microsoft Research and Carnegie
Mellon University. The first official release of Lean was announced at CADE-25
in August 2015, making it just a few months younger than Rust. As of this
writing, development on Lean is focused on the next, unreleased version that
will feature powerful automation written in Lean itself.

Lean supports two different interpretations of Martin-Löf type theory:
a purely constructive one based on Homotopy Type Theory, and one based on the
Calculus of Inductive
Constructions~\cite{coquand1988calculus,pfenning1989inductively} as championed
by the Coq~\cite{barras1997coq} theorem prover, which supports both constructive
and classical reasoning. We use the latter in our work.

The primitive type in dependent type theory is the dependent function (or
product) type $\Pi x : A, B$, where $x$ may occur in $B$; if it does not, we
obtain the standard function type $A \rightarrow B$. Function abstraction and
application extend naturally to dependent functions, as perhaps best described
by their formal typing rules.

\[
  \inferrule
    {\Gamma, x : A \vdash e : B}
    {\Gamma \vdash (\lambda x : A, e) : \Pi x : A, B}
  \hspace{4em}
  \inferrule
    {\Gamma \vdash f : \Pi x : A, B \\ \Gamma \vdash e : A}
    {\Gamma \vdash f e : [e/x]B}
\]

The Calculus of Inductive Constructions extends basic dependent type theory with
a type scheme for inductive types, which are described by a set of (possibly
dependent) functions, their \emph{constructors}. \autoref{lst:ind} shows basic
inductive definitions from the Lean standard library.

\begin{listing}[btp]
  \begin{minted}{lean}
inductive empty : Type

inductive unit : Type :=
star : unit

inductive prod (A B : Type) : Type :=
mk : A → B → prod A B

inductive sum (A B : Type) : Type :=
| inl : A → sum A B
| inr : B → sum A B

-- the dependent sum type
inductive sigma (A : Type) (B : A → Type) : Type :=
mk : Π(x : A), B x → sigma A B

inductive bool : Type :=
| ff : bool
| tt : bool

inductive option (A : Type) : Type :=
| none : option A
| some : A → option A

inductive nat : Type :=
| zero : nat
| succ : nat → nat
  \end{minted}

  \caption{The most basic inductive types as well as some basic types from
    functional programming in Lean}
  \label{lst:ind}
\end{listing}

As we can see in \autoref{lst:ind}, inductive types themselves are instances of
a type, namely \lean!Type!. This turns out to be a slight simplification,
however. More specifically, Lean has a whole hierarchy of indexed types
or \emph{universes} \lean!Type.{i}!, with \lean!Type.{i} :
Type.{i+1}!.\footnote{The hierarchy is, however, not cumulative: It is not true
  that \lean!Type.{i} : Type.{i+2}!.} The universe hierarchy is needed to avoid
the type theoretic equivalent of Russell's paradox. When we write just
\lean!Type! for the type of an inductive definition like in \autoref{lst:ind},
a correct universe level $i \ge 1$
(possibly dependent on argument universe levels) will automatically be inferred.
The reason for skipping \lean!Type.{0}! is that it has a special function that
is suggested by its more common name, \lean!Prop!: It is the universe we normally
declare types in that are to be interpreted as propositions.

Under the Curry-Howard isomorphism, an objects of a type can be interpreted as
a proof of a proposition. The reason Lean uses a separate universe for this
interpretation is that \lean!Prop! is given a specific property that would not
make sense for the other universes: By \emph{proof irrelevance}, any two
objects of a type in \lean!Prop! are definitionally equal. In other words,
proofs are irrelevant for computation. Finally, \lean!Prop! is also
\emph{impredicative}: If \lean!B : Prop!, then also \lean!(Π x : A, B) : Prop!
for any \lean!A!. This property ensures that predicates and universal
quantifications are still propositions.
The separation of inductive types and inductive propositions can lead to some duplication, which
however turns out to be very useful in ensuring suggestive names and notations
for each side (\autoref{tab:dataprop}).

\begin{table}[bt]
  \centering

  \begin{tabular}{ll}
    \toprule
    \multicolumn{2}{c}{type name (notation)}\\
    in \lean!Type! & in \lean!Prop! \\
    \midrule
    \lean!empty! & \lean!false! \\
    \lean!unit!  & \lean!true!  \\
    \lean!prod! (\lean!×!) & \lean!and! (\lean!∧!) \\
    \lean!sum! (\lean!+!) & \lean!or! (\lean!∨!) \\
    \lean!sigma! (\lean!Σ x : A, B!) & \lean!Exists! (\lean!∃ x : A, B!) \\
    \lean!Π x : A, B! & \lean!∀ x : A, B! \\
    \lean!A → B! & \lean!A → B! \\
    \bottomrule
  \end{tabular}

  \caption{The basic Curry-Howard correspondence. The table lists types from
    \autoref{lst:ind} and the corresponding types from the standard library
    with the same constructors,
    but declared in \lean!Prop!. We also show their notations as well as the
    special universal quantifier notation for dependent functions into
    \lean!Prop!. Nondependent functions and implications are not distinguished
    by notation.}
  \label{tab:dataprop}
\end{table}

On top of its interpretation of dependent type theory, Lean includes many
notational amenities. On the type level, in addition to basic and inductive
definitions, it features syntactic \lean!abbreviation!s as well as
\lean!structure!s. The latter are single-constructor inductive types that
automatically define projections to each of their constructor parameters (or
\emph{fields}) and furthermore support inheriting fields from other structures.

\begin{minted}{lean}
structure point2 :=
(x : ℕ)
(y : ℕ)

structure point3 extends point2 :=
(z : ℕ)

example : point2 := ⦃point2, x := 0, y := 1⦄
check point3.x -- point3.x : point3 → ℕ
\end{minted}

In addition to the standard parameter syntax
\lean!(x : A)!, Lean also supports two more binding modes, \lean!{x : A}!
and \lean![x : A]!.
%, with which \lean!x! will be inserted by automatic \emph{elaboration} instead
%of explicit function application.
In the first one, \lean!x! is an \emph{implicit} parameter
and will be inferred from other parameters or the expected result type, such as
in the constructor of the ubiquitous type \lean!eq! modeling Leibniz equality:

\begin{minted}{lean}
inductive eq {A : Type} (a : A) : A → Prop :=
refl : eq a a -- explicit form: @eq A a a
\end{minted}

The binding mode \lean![x : A]! instructs Lean to infer \lean!x! by
\emph{type class inference}. Type classes are arbitrary definitions annotated with the
\lean![class]! attribute. Type class inference synthesizes instances of a class
by a Prolog-like search through definitions of the class type marked with \lean![instance]!.

\begin{minted}{lean}
structure inhabited [class] (A : Type) : Type :=
(value : A)

definition default (A : Type) [inhabited A] : A :=
inhabited.value A

definition nat.is_inhabited [instance] : inhabited ℕ :=
⦃inhabited, value := 0⦄
definition prod.is_inhabited [instance] (A B : Type)
  [inhabited A] [inhabited B] : inhabited (A × B) :=
⦃inhabited, value := (default A, default B)⦄

eval default (ℕ × ℕ) -- (0, 0)
\end{minted}

In order to keep definition signatures short, we will also make use of Lean's
\lean{section} mechanism that allows us to fix common parameters for a set of definitions.

\begin{minted}{lean}
section
  -- in this section, implicit in signatures and in use sites
  parameter (A : Type)
  -- implicit in signatures but explicit in use sites
  variable (x : A)
  
  definition f : A := x
  check f -- f : A → A
end

check f -- f : Π A : Type, A → A
\end{minted}
\todo{extend ad nauseam when needed}
