\section{The Basic Transformation}
\label{sec:trans}

In this section, we describe the basic translation from Rust to Lean that
includes pure code as well as mutable local variables and loops, but not mutable
references (see~\autoref{sec:mutref}). We focus on the parts that are unique to
Rust or are nontrivial to translate. We roughly follow the structure of the
Rust Reference.\footnote{\url{https://doc.rust-lang.org/reference.html}} Because
our translation output is not optimized for readability, all sample translations
in this section have been prettified manually without changing their semantics.
An non-prettified feature-by-feature breakdown is also available
online.\footnote{http://kha.github.io/electrolysis/}

\subsection{The MIR}
\label{sec:mir}

\begin{figure}[!bp]
  \centering
  \begin{tikzpicture}
    \node (1) [block] { source };
    \node (2) [block,below=of 1] { AST };
    \node (3) [block,below=of 2] { HIR };
    \node (4) [block,below=of 3] { MIR };
    \node (5) [block,below=of 4] { LLVM IR };
    \node (6) [block,right=3cm of 4] { Lean };
    \draw[->] (1) to (2);
    \draw[->] (2) edge[loop left] node[align=right] {macro expansion\\name resolution} (2);
    \draw[->] (2) to (3);
    \draw[->] (3) edge[loop left] node[align=right] {lifetime resolution\\validity checks} (3);
    \draw[->] (3) to (4);
    \draw[->] (4) edge[loop left] node[align=right] {optimizations\\borrow checking} (4);
    \draw[->] (4) to (5);
    \draw[->] (4) to node {our work} (6);
  \end{tikzpicture}
  \caption{Overview of the Rust compiler pipeline and our work in that context}
  \label{fig:mir}
\end{figure}

Because Rust makes extensive use of inference algorithms for types, lifetimes and typeclasses,
correctly parsing Rust code is no small feat. Therefore, we use the Rust
Compiler \texttt{rustc} itself as a frontend and work on the completely explicit and
much simpler \emph{mid-level intermediate representation} (MIR)
(\autoref{fig:mir}). By writing our translation program in Rust, we can import
the \texttt{rustc} libraries to gain access to the MIR and many convenient
helper functions.

The MIR is a control flow graph (CFG) representation where a basic block consists
of a list of statements followed by a terminator that (conditionally or
unconditionally) transfers control to other basic blocks. For readability,
this section will mostly argue on the Rust source level, but the graph structure
will be important for translating control flow.

%Apart from annotation statements and ones that will be inserted in the backend, the only statement kind we have to handle is the assignment of an rvalue to an lvalue. 

\subsection{Identifiers}

Working on top of the MIR, we do not have to worry about the lexical structure
of Rust. We do, however, have to make sure we emit lexically correct Lean code.
This is only a problem with identifiers, which we would like to transfer with
minimal changes. Both languages are based on segmented identifiers, just with
different separators (\rust{a::b::c} in Rust versus \lean{a.b.c} in Lean).
However, some identifier parts in Rust such as \rust{[T]} or \rust{<F<T> as S>}
are not valid in Lean. To retain readability, we have therefore extended Lean
with a general escaping syntax for identifiers that allows arbitrary symbols by
surrounding them with \lean{«} and \lean{»}: The identifier \lean{«[T]».«a.b»}
is now a valid Lean identifier consisting of the parts \lean{«[T]»} and
\lean{«a.b»}.

\subsection{Programs and Files}

Rust's unit of compilation is called a \emph{crate}. A crate consists of one or
more \texttt{.rs} files and can be compiled to an executable or library. Files
inside a crate may freely reference declarations between them. On the other
hand, Lean files may only import other files non-recursively and declarations
must be strictly sorted in order of usage for termination checking. We therefore
translate a crate into a single Lean file and perform a topological sort on its
declarations. While Lean does support explicit declarations of mutually
recursive types and functions, we have not yet encountered such declarations in
Rust code as part of our formalization work and thus have not implemented support for them so far.

In detail, our tool creates a file called \verb!generated.lean! in a separate
folder for each crate and connects them using Lean's \lean{import} directive
according to the inter-crate dependencies. The user can additionally create a
\verb!pre.lean! file that will automatically be imported and can be used for
axiomatizations as well as a \verb!config.toml! file that can influence the
translation (see below for examples). We use a third Lean file \verb!thy.lean! per crate
for the proofs, which will import both the generated code and proof files from
other crates.

\subsection{Types}

\subsubsection{Primitive Types}
\label{sec:prim}

Rust's primitive types are the boolean type, machine-independent and machine-dependent integer
types, tuples, arrays, slices, and function types.

Following AutoCorres' design (see \autoref{sec:related}), we map the primitive integer types to
Lean's native arbitrary-sized types and instead handle overflow explicitly
during computation (\autoref{sec:arith}).

\begin{minted}{lean}
abbreviation u8 [parsing_only] := nat
abbreviation u16 [parsing_only] := nat
abbreviation u32 [parsing_only] := nat
abbreviation u64 [parsing_only] := nat
abbreviation usize [parsing_only] := nat

abbreviation i8 [parsing_only] := int
// ...

definition u8.bits [reducible] : ℕ := 8
// ...

definition usize.bits : ℕ := 16
lemma usize.bits_ge_16 : usize.bits ≥ 16 := dec_trivial
attribute usize.bits [irreducible]
\end{minted}

For the machine-size integer types \rust{usize} and \rust{isize}, we only expose
the conservative assumption that their bit sizes are at least 16. We still define
\rust{usize.bits} to be exactly 16 so that it is computable, but by then marking
the definition as \rust{[irreducible]}, this fact is only accessible in proofs
when explicitly unfolding the definition.
When a proof does rely on the bounds of a parameter, we can add a separate
hypothesis, for which we make use of typeclasses. The bounds of an expression
can often be determined just from partial information, such as with unsigned division.

\begin{minted}{lean}
definition is_bounded_nat [class] (bits x : ℕ) := x < 2^bits
abbreviation is_usize := is_bounded_nat usize.bits

lemma div_is_bounded_nat [instance] (bits x y : ℕ)
  [is_bounded_nat bits x] : is_bounded_nat bits (x / y) := ...
\end{minted}

We use the same approach for arrays (\rust{[T; N]}) and slices (\rust{&[T]}),
mapping both the to arbitrary-length \lean{list} type. While Rust arrays have a
constant length encoded in the type, slices are dynamic views into contiguous sequences
like arrays or \rust{Vec}s and bounded only by the memory size. More
specifically, they (and any Rust type) are assumed to be no larger than
\rust{isize::MAX} bytes so that the pointer difference of any two elements can
be represented by an \rust{isize} value.

\begin{minted}{lean}
abbreviation array [parsing_only] (A : Type₁) (n : ℕ) := list A
abbreviation slice [parsing_only] := list

definition is_slice [class] {A : Type₁} (xs : slice A) :=
length xs < 2^(usize.bits-1)
\end{minted}

\subsubsection{Structs and Enums}

Because Rust does not feature inheritance, struct types and enumerated types are
true Algebraic Data Types and can directly be translated to their Lean
equivalents (\lean{structure} and \lean{inductive}, respectively).

\subsubsection{References}
\label{sec:refs}

An immutable reference \rust{&'a T} is checked by the Rust compiler not to alias
with any mutable reference and thus can be safely replaced with the translation
of \rust{T} itself. We drop all lifetime specifiers in general because we trust
the Rust compiler to already have made all memory safety checks.

We will discuss mutable references in \autoref{sec:mutref}.

\subsection{Traits}

Rust's trait system is similar to Haskell's type classes, but borrows some
syntax from more object-oriented \emph{interface} systems. In particular, in addition
to functions a trait may also contain methods that can be called on any object
of a type the trait is implemented \emph{on}. This is implemented via an implicit
type parameter \rust{Self} that is used for the type of the \lean{self}
parameter and is specified in the \rust{for} clause when implementing a trait via an \rust{impl} block.

\begin{figure}[bt]
\begin{sbs1}
struct S { i: i32 }

trait Trait<T> {
  fn f(self) -> T;
}

impl Trait<i32> for S {
  fn f(self) -> i32 {
    self.i
  }
}

fn g<T : Trait<i32>>
  (t: T) -> i32 {
  t.f()
}

fn h() -> i32 {
  g(S { i: 0 })
}
\end{sbs1}
\begin{sbs2}
structure S := (i : i32)

structure Trait [class] (Self T : Type₁) :=
(f : Self → sem T)

definition «S as Trait<i32>».f (self : S) : sem i32 :=
return (S.i self)

definition «S as Trait<i32>» [instance] :=
⦃Trait S i32, f := «S as Trait<i32>».f⦄

definition g {T : Type₁} [Trait T i32] (t : T) : sem i32 :=
sem.incr 1 (Trait.f _ t)

definition h : sem i32 :=
sem.incr 1 (g (S.mk 0))
\end{sbs2}

\caption{A parametric trait in Rust and its translation.}
\label{fig:trait}
\end{figure}

The translation of basic traits into Lean type classes is straightforward
(\autoref{fig:trait}). We will discuss the details of the function-level translation and the \lean{sem}
monad below. While \rust{impl} blocks in Rust are anonymous, we need to name all
definitions in Lean and do so using a naming scheme similar to \texttt{rustc}'s
own internal representation.

\subsubsection{Default Methods}

As shown in \autoref{fig:trait}, we generate separate definitions for functions in trait
implementations before assembling them into a type class instance. This way, and
by eliminating the type class indirection in calls to a statically known
implementation, we can allow trait implementation functions to call each other
using our standard topological dependency ordering.

\begin{sbs1}
struct S;

trait Trait {
  fn f(self);
  fn g(self);
}

impl Trait for S {
  fn f(self) {
    self.g()
  }
  fn g(self) {}
}
\end{sbs1}
\begin{sbs2}
structure S := (i : i32)

structure Trait [class] (Self : Type₁) :=
(f : Self → sem unit)
(g : Self → sem unit)

definition «S as Trait».g (self : S) : sem unit :=
return unit.star

definition «S as Trait».f (self : S) : sem unit :=
sem.incr 1 («S as Trait».g self)

definition «S as Trait» [instance] :=
⦃Trait S, f := «S as Trait».f, g := «S as Trait».g⦄
\end{sbs2}

However, just like Haskell, Rust also allows default implementations of trait
methods that may arbitrarily call and be called from other trait methods that
will only be defined in some implementation of the trait later on. This
makes static ordering of dependencies impossible.

In essence, a default method in a trait takes as input an instance of that trait
to call other trait methods with, but at the same time has to be a slot in the
very same trait because it may be overridden in an implementation.

There are multiple potential ways to deal with that depdendency cycle. We could
simply create a specialized copy of the trait method for each instantiation, but
then we would also have to copy proofs about it. We could try to dynamically solve the cycle in a general way, computing its least fixed point
by use of the Knaster-Tarski theorem~\cite{tarski1955lattice} as usual in
denotational semantics. Or we can restrict ourselves to special cases that break
the cycle. If we remove the trait instance as an input to the default method, it
cannot call other trait methods. If, on the other hand, we do not make default
methods part of the trait instance, it
cannot be overridden or be called from inside implementations of the trait. We
could even mix these two approaches, incrementally building up the
trait instance by alternating between default and non-default methods.

We could implement all these approaches and automatically or manually choose
between them on a case-by-case basis. It turns out, however, that in the Rust standard library, default methods are often just
convenience wrappers around other trait methods, like in the \rust{PartialEq} trait.

\begin{minted}{rust}
pub trait PartialEq<Rhs> {
  fn eq(&self, other: &Rhs) -> bool;
  fn ne(&self, other: &Rhs) -> bool { !self.eq(other) }
}
\end{minted}

Therefore, as of now we have only implemented the third approach of declaring
default methods outside of their trait, which turned out to be sufficient for
our verification work so far.

\begin{minted}{lean}
structure PartialEq [class] (Self Rhs : Type₁) :=
(eq : Self → Rhs → sem bool)

definition PartialEq.ne {Self Rhs : Type₁} [PartialEq Self Rhs] (self : Self) (other : Rhs) : sem bool :=
do t5 ← sem.incr 1 (PartialEq.eq self other);
return (bool.bnot t5)
\end{minted}

\subsubsection{Associated Types}

There is one further advanced trait feature Rust shares with Haskell called
\emph{associated types}: trait members that are not functions, but types.

\begin{minted}{rust}
pub trait Add<RHS> {
  type Output;
  fn add(self, rhs: RHS) -> Output;
}
\end{minted}

Making \rust{Output} an associated type instead of a type parameter
fundamentally changes type class inference: Instead of being an input parameter
to the inference like \rust{Self} and \rust{RHS}, \rust{Output} is
\emph{determined} by the inferred trait instance. This means that inference on
\rust{add} can succeed even if the expected return type is unknown.

As a dependently typed language, Lean has no problem with representing such
traits as type classes. What it cannot represent, however, is a special class
of trait bounds Rust supports: \rust{T : Add<RHS, Output=RHS>} asserts a
definitional equality on the associated type; but definitional equality exists
only as a judgment in Lean, not as a proposition we could pass as a parameter.
Instead, we follow the
original paper~\cite{chakravarty2005associated} on associated types in Haskell
that translates type classes with associated types into System F by turning them
into type parameters.

\begin{minted}{lean}
structure Add [class] (Self RHS Output : Type₁) :=
(add : Self → RHS → sem Output)
\end{minted}

This transformation does weaken type class inference, which means that in the
generated Lean code, we have to resort to passing type class
arguments explicitly using the \lean{@} notation. We might be able to regain
inference in a potential future version of Lean that supports functional dependencies~\cite{jones2000type}.

\subsubsection{Trait Objects}

Lastly, Rust's trait system exhibits a feature that does not directly exist in
Haskell. In Haskell, type classes are not types - they cannot explicitly be passed by
value, only implicitly through inference. In Rust, traits are dynamically sized
types, which means they can be used as values, but only behind some indirection like
\rust{&Trait}. These \emph{trait objects} are represented as a pointer to a
vtable of the trait implementation and another pointer to the \emph{Self} value.

This ``fat pointer'' representation would translate quite naturally to an existential type
\lean{Σ (Self : Type), (Trait Self × Self)}. What is not apparent in this
natural definition, however, is the fact that it necessarily lives in a higher universe
than \lean{Self}. This is the only construct currently in Rust that can give rise to a type
not in \lean{Type₁} (but, in fact, to a type in an arbitrarily high universe through
nesting). It is an open problem in the Lean community if and how a monad over
types of different universes can cleanly work given Lean's non-cumulative
universe hierarchy. Fortunately, trait objects are a rare feature in Rust code
that we do not expect to find on the algorithmical level of our current
verification work, so we have not investigated this issue any further for now.

\subsection{The Semantics Monad}

The core part for representing Rust's dynamic semantics is the monadic embedding. While
higher-order unification issues in the current Lean version prevent us from
outright parameterizing the embedding by an arbitrary monad instance, we still
try to keep the interface of our specific monad abstract so that the monad can be
extended in the future.

We currently model abnormal termination\footnote{unspecified behavior like integer
overflow and \emph{panics} from out-of-bounds array accesses or explicit \rust{panic!}
calls. Rust does not have exceptions.} and
nontermination as well as an abstract step counter for asymptotic run time analysis.

\begin{minted}{lean}
definition sem (A : Type₁) := option (A × ℕ)
\end{minted}

We provide the standard monadic operations on the type, including a
\texttt{do}-notation. The model-specific operations are \lean{mzero}
indicating abnormal termination/nontermination, and \lean{sem.incr}, which
increments the step counter (if any). An increment of one is emitted around
every Rust function call and before each loop iteration.

\begin{minted}{lean}
definition mzero {A : Type₁} : sem A := none
definition return {A : Type₁} (x : A) : sem A := some (x, 0)

definition sem.incr {A : Type₁} (n : ℕ) : sem A → sem A
| (some (x, k)) := some (x, k+n)
| none          := none

definition sem.bind {A B : Type₁} (m : sem A) (f : A → sem B)
  : sem B :=
option.bind m (λs, match s with
| (x, k) := sem.incr k (f x)
end)

infixl ` >>= `:2 := sem.bind
\end{minted}

The semantics monad follows the usual monad laws, which we will make use of in proofs.

\begin{minted}{lean}
lemma return_bind {A B : Type₁} {a : A} {f : A → sem B}
  : (return a >>= f) = f a := ...
lemma bind_return {A : Type₁} {m : sem A} : (m >>= return) = m := ...
lemma bind.assoc {A B C : Type₁} {m : sem A} {f : A → sem B}
  {g : B → sem C} : (m >>= f >>= g) = (m >>= (λx, f x >>= g)) := ...
\end{minted}

\subsection{Statements and Control Flow}

The local state of a Rust function consists of its arguments, variables, and
temporaries (variables introduced by the compiler). Without mutable references,
these locals can only be manipulated by assignments, the single statement kind
available in the MIR. In linear code, keeping track of assignments is as easy as
transforming them to redeclarations.

\begin{sbs1}
p.x += 1;
\end{sbs1}
\begin{sbs2}
let p = Point { x = p.x + 1, ..p };
\end{sbs2}

Nonlinear control flow is introduced by Rust's \rust{if} and \rust{match}
constructs as well as its three loop constructs (which have a single common
representation in the MIR). We map the first two cases to Lean's corresponding
constructs of the same names.

\vspace{1em}\noindent\begin{minipage}[t]{0.3\textwidth}
  \begin{minted}{rust}
let x = if b {1} else {0};
x & 1
  \end{minted}
\end{minipage}
\hfill\vline\hfill\begin{adjustbox}{valign=t,minipage={0.33\textwidth}}
\begin{tikzpicture}
  \node (1) [block] { \rust{if b} };
  \node (2) [block,below=of 1,xshift=-9mm] { \rust{x = 0;} };
  \draw[->] (1) to node[left] {\rust{false}} (2);
  \node (3) [block,below=of 1,xshift=9mm] { \rust{x = 1;} };
  \draw[->] (1) to node[right] {\rust{true}} (3);
  \node (4) [block,below=of 2,xshift=9mm] { \rust{ret = x & 1; return} };
  \draw[->] (2) to (4);
  \draw[->] (3) to (4);
\end{tikzpicture}
\end{adjustbox}
\hfill\vline\hfill\begin{minipage}[t]{0.3\textwidth}
  \begin{minted}{lean}
if b = bool.tt then
  let x := 1 in
  x & 1
else
  let x := 0 in
  x & 1
  \end{minted}
\end{minipage}\vspace{1em}

As can be seen, we currently translate each branch of a conditional block
terminator independently, which can lead to code duplication if those branches
converge again. While this has not manifested any problems in our verification
work so far, we may want to mitigate it in the future by factoring out the
common translated code into a separate definition.

\begin{figure}[!b]
\hspace{1cm}\begin{minipage}[t]{0.4\textwidth}
  \begin{minted}{rust}
fn f() {
  let mut x = 0;
  while x < 10 {
      x += 1;
  }
}
  \end{minted}
\end{minipage}
\hfill\vline\hfill\begin{adjustbox}{valign=t,minipage={0.4\textwidth}}
  \newsavebox{\mintedbox}
  \begin{lrbox}{\mintedbox}
    \begin{minipage}[t]{1.8cm}
\begin{minted}{rust}
x = 0;
if x < 10
\end{minted}
    \end{minipage}
  \end{lrbox}
\begin{tikzpicture}
  \node (start) {};
  \node (1) [block,label=left:1,below=5mm of start] {\usebox{\mintedbox}};
  \draw[->] (start) to (1);
  \node (2) [block,label=left:2,below=of 1,xshift=-15mm] { \rust{return} };
  \draw[->] (1) to node[left] {\rust{false}} (2);
  \node (3) [block,label=left:3,below=of 1,xshift=15mm] { \rust{x = x + 1;} };
  \draw[->] (1) to node[right] {\rust{true}} (3);
  \draw[->] (3) to[bend left=45] (1);
\end{tikzpicture}
\end{adjustbox}

\caption{A \rust{while} loop and the corresponding (simplified) MIR graph.
  Blocks 1 and 3 from a strongly connected component, which is dominated by
  block 1, the loop header.}
\label{fig:scc}

\end{figure}

We do need to factor out common code in the case of loops. There is no special
terminator signifying loops in the MIR; instead, we have to search for
(nontrivial) strongly connected components (SCCs) of basic blocks (\autoref{fig:scc}). Because Rust's
control flow is \emph{reducible} (notably, lacking a \emph{goto} instruction),
we may assume that such an SCC can only be entered from a single node
(\emph{dominating} the SCC). With this, we can describe the semantics of the SCC
in more traditional terms of iteration: The dominating node is the \emph{loop
  header}, while the rest of the SCC is the \emph{body}. Jumping back to the
header signifies a new iteration, while jumping out of the SCC means breaking
the loop. By breaking up the SCC at the header, we can thus translate a single
iteration to a function of type

\begin{minted}{lean}
State → sem (State + Res)
\end{minted}
that takes a tuple \lean{State} of loop variables and either returns the new
state for the next iteration, or a value of the source function's return type
\lean{Res} when breaking out of the loop. We tie this into a single value of
type \lean{sem Res} by use of a general \emph{loop combinator}.

\subsubsection{The Loop Combinator}
\label{sec:loop}

The loop combinator has the signature

\begin{minted}{lean}
noncomputable definition loop {State Res : Type₁}
  (body : State → sem (State + Res)) (s : State) : sem Res
\end{minted}
Its task is to apply \rust{body} repeatedly (starting with \rust{s}) until some
\rust{Res} is returned; if the loop does not terminate, it returns \rust{mzero}
(which \rust{body} may also return by itself). Termination for arbitrary values
of \rust{body} obviously is not a decidable property. Therefore we will have to leave
the constructive subset of Lean, as signified by the \lean{noncomputable}
specifier. The (simplified) translation of the Rust code in \autoref{fig:scc} via
\lean{loop} is as follows:

\begin{minted}{lean}
definition f.loop_1 (x : i32) : sem (i32 + unit) :=
if x < 10 then
  let x := x + 1 in
  return (sum.inl x)
else
  return (sum.inr unit.star)

definition f : sem unit :=
let x := 0 in
loop f.loop_1 x
\end{minted}

As a total, purely functional language, Lean cannot express iteration directly,
and the only primitive kind of recursion available in Lean is structural recursion
over an inductive datatype. On top of structural recursion, the Lean standard
library defines the more general concept of \emph{well-founded} recursion: A
relation \lean{R : A → A → Prop} on a type \lean{A} is well-founded if every
element of \lean{A} is \emph{accessible} through the relation, which is defined
inductively as all predecessors of the element under the relation being accessible.

\begin{minted}{lean}
inductive acc {A : Type} (R : A → A → Prop) : A → Prop :=
intro : ∀x, (∀ y, R y x → acc R y) → acc R x

inductive well_founded [class] {A : Type} (R : A → A → Prop) : Prop :=
intro : (∀ a, acc R a) → well_founded R
\end{minted}

Using structural recursion over the \lean{acc} predicate, the standard library
defines a fixed-point combinator for functionals respecting a well-founded
relation, and proves that the combinator satisfies the fixpoint equation.

\begin{minted}{lean}
namespace well_founded
section
  variables {A : Type} {C : A → Type} {R : A → A → Prop}

  definition fix [well_founded R] (F : Πx, (Πy, R y x → C y) → C x)
    (x : A) : C x := ...

  theorem fix_eq [well_founded R] (F : Πx, (Πy, R y x → C y) → C x)
    (x : A) : fix F x = F x (λy h, fix F y) := ...
end
end well_founded
\end{minted}

We use well-founded recursion to define \lean{loop}: If repeatedly applying
\emph{body} to \emph{s} yields a sequence of states,
this sequence will terminate iff there exists a well-founded relation on
\lean{State} such that the sequence is a descending chain.
This is true because descending chains in well-founded relations are finite, and
conversely a finite sequence $s_1 = s, \dots, s_n$ is a descending chain in the
trivial well-founded relation $R = \{(s_{i+1}, s_i) | 1 \le i < n\}$.

In the formalization, given a well-founded relation \lean{R} on \lean{State}, we first have to take care of lifting it to a well-founded relation \lean{R'} on \lean{State + Res}.

\begin{minted}{lean}
section
  parameters {State Res : Type₁}
  parameter (body : State → sem (State + Res))
  parameter (R : State → State → Prop)

  definition State' := State + Res

  definition R' : State' → State' → Prop
  | (inl s') (inl s) := R s' s
  | _        _       := false

  private lemma R'.wf [instance] [well_founded R] : well_founded R' := ...
\end{minted}

We can then wrap \lean{body} in a functional respecting \lean{R'} that we can pass to \lean{well_founded.fix}.

\begin{minted}{lean}
  definition F (x : State') (f : Π (x' : State'), R' x' x → sem State') : sem State' :=
  match x with
  | inr _ := mzero -- unreachable
  | inl s :=
    do x' ← sem.incr 1 (body s);
    match x' with
    | inr r := return (inr r)
    | x'    := if H : R' x' x then f x' H else mzero
    end
  end

  definition loop.fix [well_founded R] (s : State) : sem Res :=
  do x ← well_founded.fix F (inl s);
  match x with
  | inr r := return r
  | inl _ := mzero -- unreachable
  end
\end{minted}

Finally, we implement \lean{loop} by choosing any well-founded relation \lean{R} that makes
the loop terminate, if any, or else return \lean{mzero}.

\begin{minted}{lean}
  definition terminating (s : State) :=
  ∃ Hwf : well_founded R, loop.fix s ≠ mzero

  noncomputable definition loop (s : State) : sem Res :=
  if Hex : ∃ R, terminating R s then
    @loop.fix (classical.some Hex) _ (classical.some (classical.some_spec Hex)) s
  else mzero
\end{minted}

Here we make use of the \emph{dependent if-then-else} notation that allows us to
test for a property and then bind a name to a proof of it in case it holds. We
then destructure that proof object to obtain the relation and its
well-foundedness proof so that we can pass them to \lean{loop.fix}. The
\lean{classical.some} and \lean{classical.some_spec} definitions are based on
Hilbert's epsilon operator.

\begin{minted}{lean}
noncomputable definition classical.some {A : Type} {P : A → Prop} (H : ∃x, P x) : A := ...
theorem classical.some_spec {A : Type} {P : A → Prop} (H : ∃x, P x) : P (some H) := ...
\end{minted}

The use of \lean{classical.some} as well as the undecidable conditional
\lean{∃ R, terminating R s} make \lean{loop} non-computable.

When verifying loops, we will first verify the corresponding application of
\lean{loop.fix} using a specific well-founded relation, for which we can prove a
convenient fixpoint equation.

\begin{minted}{lean}
theorem loop.fix_eq
  {R : State → State → Prop} [well_founded R] {s : State} :
  loop.fix R s =
    do x' ← sem.incr 1 (body s);
    match x' with
    | inl s' := if R s' s then loop.fix R s' else mzero
    | inr r  := return r
    end := ...
\end{minted}

If the application of \lean{loop.fix} terminates, we can show that the original
application \lean{loop} will do so too with the same return value,
via a helper lemma that says that all terminating \lean{loop.fix} applications are equal.

\begin{minted}{lean}
lemma loop.fix_eq_fix
  {R₁ R₂ : State → State → Prop} [well_founded R₁] [well_founded R₂]
  {s : State}
  (Hterm₁ : loop.fix R₁ s ≠ mzero)
  (Hterm₂ : loop.fix R₂ s ≠ mzero) :
  loop.fix R₁ s = loop.fix R₂ s := ...

theorem loop.fix_eq_loop
  {R : State → State → Prop} [well_founded R]
  {s : State}
  (Hterm : loop.fix R s ≠ mzero) :
  loop.fix R s = loop s := ...
\end{minted}

\subsection{Expressions}

\subsubsection{Arithmetic Operators}
\label{sec:arith}

Rust's arithmetic semantics is based on the premise that in most circumstances,
arithmetic overflow is unintended by the programmer,\footnote{When overflowing
  is indeed intended, the programmer may use special methods such as
  \rust{u8::wrapping_add}} and constitutes a bug in
the program. Therefore, in debug builds, the built-in arithmetic operators will
panic on any overflow. In release builds, overflows for both signed and unsigned
types will wrap for performance reasons.

We thus regard arithmetic overflow in those operators as \emph{unspecified} and
return the bottom value in such cases, using the predicate
\lean{is_bounded_nat} from \autoref{sec:prim}.

\begin{minted}{lean}
definition sem.guard {a : Type₁} (p : Prop) [decidable p] (s : sem a) : sem a :=
if p then s else mzero

definition check_unsigned (bits : ℕ) (x : nat) : sem nat :=
sem.guard (is_bounded_nat bits x) (return x)

definition checked.add (bits : ℕ) (x y : nat) : sem nat :=
check_unsigned bits (x+y)

...
\end{minted}

We can avoid the check in operations that cannot overflow, such as unsigned
division. We still have to check for division by zero, of course.

\begin{minted}{lean}
definition checked.div (bits : ℕ) (x y : nat) : sem nat :=
sem.guard (y ≠ 0) (return (div x y))
\end{minted}

The signed implementations are equivalent, except that we do have to check for
overflow during signed division by $-1$.

\subsubsection{Bitwise Operators}

We implement all bitwise operations on integral types by converting them to and
from the \lean{bitvec} type, which we adapted from the Lean standard
library and expanded significantly.

\begin{minted}{lean}
abbreviation binary_bitwise_op (bits : ℕ) (op : bitvec bits → bitvec bits → bitvec bits)
  (a b : nat) : nat :=
bitvec.to ℕ (op (bitvec.of bits a) (bitvec.of bits b))

definition bitor bits := binary_bitwise_op bits bitvec.or
...
\end{minted}

Some care must be taken when implementing bitwise shift: Shifting by the bitsize
of a type or more bits will panic in Rust.

\begin{minted}{lean}
definition checked.shl [reducible] (bits : ℕ) (x : nat) (y : u32) : sem nat :=
sem.guard (y < bits)
  (return (bitvec.to ℕ (bitvec.shl (bitvec.of bits x) y)))
\end{minted}

\subsubsection{Index Expressions}

While indexing is desugared to a call to the \rust{Index} trait for most types,
it is a primitive operation on arrays. Out-of-bounds accesses will panic in
Rust. By identifying arrays with Lean lists, we can use the existing
\lean{list.nth} function and \emph{lifting} its result into the semantics monad.

\begin{minted}{lean}
definition sem.lift_opt {A : Type₁} : option A → sem A
| none   := mzero
| some a := return 
\end{minted}

\subsubsection{Lambda Expressions}
\label{sec:lambda}

Each lambda expression in Rust has a unique type that represents its
\emph{closure}, the set of variables captured from the outer scope.
As necessitated by its ownership and mutability tracking, Rust files each
closure type into one of three traits that together form a hierarchy:

\begin{minted}{rust}
pub trait FnOnce<Args> {
  type Output;
  fn call_once(self, args: Args) -> Output;
}

pub trait FnMut<Args> : FnOnce<Args> {
  fn call_mut(&mut self, args: Args) -> Output;
}

pub trait Fn<Args> : FnMut<Args> {
  fn call(&self, args: Args) -> Output;
}
\end{minted}

Rust will automatically infer the most general trait based on the lambda
expression's requirements: If it has to move ownership of a captured variable, it
can only implement \rust{FnOnce}; if it needs a mutable reference to a variable,
it can only implement \rust{FnMut}; otherwise, it will implement \rust{Fn}.

Because we lose the restrictions of linear typing during our translation, we can
simplify the hierarchy: \rust{FnOnce} can be implemented using
\rust{FnMut}, if implemented, which in turn can be implemented using
\rust{Fn} (because the closure must be immutable in that case).

\begin{minted}{lean}
structure FnOnce [class] (Self Args Output : Type₁) :=
(call_once : Self → Args → sem Output)

structure FnMut [class] (Self Args Output : Type₁) :=
(call_mut : Self → Args → sem (Output × Self))

definition FnMut_to_FnOnce [instance] (Self Args Output : Type₁)
  [FnMut Self Args Output] : FnOnce Self Args Output :=
⦃FnOnce, call_once := λ self args, do x ← FnMut.call_mut _ self args;
   return x.1⦄

structure Fn [class] (Self : Type₁) (Args : Type₁) (Output : Type₁) :=
(call : Self → Args → sem Output)

definition Fn_to_FnMut [instance] (Self Args Output : Type₁) [Fn Self Args Output]
  : FnMut Self Args Output :=
⦃FnMut, call_mut := λ self args, do x ← Fn.call _ self args;
  return (x, self)⦄
\end{minted}

Translating a lambda expression means declaring a closure type according to the
captured environment and creating a trait implementation according to the
closure kind as reported by the compiler. Calling a lambda expression, on the
other hand, is no different from other trait method calls and does not need any
special casing.

\begin{sbs1}
fn f(x: i32) -> i32 {
  let l = |y| x + y;
  l(x)
}
\end{sbs1}
\begin{sbs2}
structure f.closure_7 := (val : i32)

definition f.closure_7.fn (c : f.closure_7) (y : i32) : sem i32 :=
let t4 := f.closure_7.val c in
checked.sadd i32.bits t4 y

definition f.closure_7.inst [instance] : Fn f.closure_7 i32 i32 :=
⦃Fn, call := f.closure_7.fn⦄

definition f (x : i32) : sem i32 :=
let l := f.closure_7.mk x in
sem.incr 1 (Fn.call _ l x)
\end{sbs2}
