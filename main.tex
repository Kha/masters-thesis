\documentclass[12pt,a4paper,twoside]{article}

%\usepackage[utf8]{inputenc}
\usepackage[T1]{fontenc}
\usepackage[english]{babel}
\usepackage{lmodern}
\usepackage{tikz}
\usepackage{fontspec}
%\setmonofont{Inconsolata}
\setmonofont[Scale=0.85]{Source Code Pro}

\usetikzlibrary{shapes.geometric,positioning,calc,arrows}

\usepackage{listings}

\usepackage{amsmath, mathpartir}
\usepackage[labelformat=simple]{subfig}
\usepackage[font={sf},margin=10pt,labelfont=bf]{caption}
\usepackage{booktabs}
\usepackage[colorlinks=false]{hyperref}
\providecommand*{\listingautorefname}{Listing}

\usepackage{newunicodechar}
\newfontfamily{\freeserif}[Scale=MatchLowercase]{DejaVu Sans}
\newunicodechar{ℕ}{\freeserif{ℕ}}
\newunicodechar{ₐ}{\freeserif{ₐ}}
\newunicodechar{₁}{\freeserif{₁}}
\newunicodechar{∈}{\freeserif{∈}}
\newunicodechar{𝓞}{\ensuremath{\mathcal{O}}}
\newunicodechar{∉}{\freeserif{∉}}
%\newunicodechar{Π}{\freeserif{Π}}
%\newunicodechar{→}{\freeserif{→}}
\newunicodechar{⦃}{\freeserif{⦃}}
\newunicodechar{⦄}{\freeserif{⦄}}
\newunicodechar{∧}{\freeserif{∧}}
\newunicodechar{∨}{\freeserif{∨}}
\newunicodechar{⊢}{\freeserif{⊢}}
\newunicodechar{⊑}{\freeserif{⊑}}
\newunicodechar{ₚ}{\freeserif{ₚ}}

\usepackage{fancyhdr}
\setlength{\headheight}{15pt}
\usepackage{todonotes, rotating, minted}
%workaround to remove red boxes
\renewcommand{\fcolorbox}[4][]{#4}
\usemintedstyle{tango}
\usemintedstyle[final]{bw}
\newmintinline[rust]{rust}{}
\newmintinline[lean]{lean}{}
\setminted{breaklines=true, fontsize=\small\linespread{0.85}}
\setmintedinline{fontsize=\normalsize}

\clubpenalty = 10000
\widowpenalty = 10000 \displaywidowpenalty = 10000

\begin{document}

\def\sectionautorefname{Section}
\def\subsectionautorefname{Subsection}
\def\subsubsectionautorefname{Subsection}

\newcommand\ie{i.e.\ }
\newcommand\eg{e.g.\ }

  \begin{titlepage}
    \begin{tikzpicture}[remember picture,overlay]
      % Seitenrahmen zeichnen.
      \draw[semithick,rounded corners=0.5cm]
        ($(current page.north west) + ( 1cm,-1cm)$) --
        ($(current page.north east) + (-1cm,-1cm)$) --
        ($(current page.south east) + (-1cm, 1.5cm)$);

      \draw[semithick,rounded corners=0.5cm]
        ($(current page.south east) + (-1cm, 1.5cm)$) --
        ($(current page.south west) + ( 1cm, 1.5cm)$) --
        ($(current page.north west) + ( 1cm,-1cm)$);

      % Logo einbinden.
      \node[anchor=north west] (logo) at ($(current page.north west) + (1.75cm,-1.5cm)$)
      {
        \includegraphics[width=4cm]{KITLogo}
      };

      % Institut / Lehrstuhl.
      \node[anchor=east] at ($(current page.east |- logo.east) + (-1.75cm,0cm)$)
      {
        \begin{minipage}{5.2cm}
          \begin{flushleft}
            \footnotesize{}Institut für Programmstrukturen und Datenorganisation (IPD) \\
            \vspace{6pt}
            Lehrstuhl Prof.\ Dr.-Ing.\ Snelting
          \end{flushleft}
        \end{minipage}
      };

      \node (title) at ($(current page.center |- logo.south) + (0cm, -4cm)$)
      {
        % Korrekter Zeilenabstand etc. durch Minipage.
        \begin{minipage}{12cm}
          \begin{center}
            \huge\textbf{Simple Verification of Rust Programs via Functional Purification}
          \end{center}
        \end{minipage}
      };

      \node[below=1.75cm of title.south]   (prename)  { Masterarbeit von };
      \node[below=0.75cm of prename.south] (name)     { \Large{}\textbf{Sebastian Ullrich} };
      \node[below=1cm    of name.south]    (postname) { an der Fakultät für Informatik };

      \node[below=3cm    of name.south]    (bildchen) { \includegraphics[width=0.9\textwidth]{logo.png}
                                                      };

      \node[below=2cm of bildchen.south] (table)
      {
        \begin{tabular}{ll}
          \textbf{Erstgutachter:}           & Prof.\ Dr.-Ing.\ Gregor Snelting \\[5pt]
          \textbf{Zweitgutachter:}          & ??? \\[5pt]
        \end{tabular}
      };

      \node[below=3.5cm of table.south] (time)
      {
        \begin{tabular}{ll}
        \textbf{Bearbeitungszeit:} & 4. Juli 2013 -- 29. Oktober 2013
        \end{tabular}
      };

      % Fußzeile, unten zentriert.
      \node[anchor=south] (footnote) at ($(current page.center |- current page.south) + (0cm, 0.65cm)$)
      {
        \tiny{}KIT -- Universität des Landes Baden-Württemberg und nationales Forschungszentrum in der Helmholtz-Gemeinschaft
        \hspace{0.5cm}
        \Large{}\textbf{www.kit.edu}
      };
    \end{tikzpicture}
  \end{titlepage}

% sane default for proof documents
%\parindent 0pt\parskip 0.5ex

\tikzset{every node/.style={transform shape},auto,block/.style={align=center,rectangle,draw,minimum height=20pt,minimum width=30pt},>=triangle 60}
%\pagenumbering{Roman}
\pagestyle{empty}
\renewcommand{\abstractname}{Einfache Verifikation von Rust-Programmen}
\begin{abstract}
  Imperative Programmiersprachen sind in der modernen Softwareentwicklung
  allgegenwärtig, stellen aber ein Hindernis für formale Softwareverifikation
  dar durch ihre Verwendung von veränderbaren Variablen und Objekten. Programme
  in diesen Sprachen können normalerweise nicht direkt auf die unveränderliche
  Welt von Logik und Mathematik zurückgeführt werden, sondern müssen in eine
  explizit modellierte Semantik der jeweiligen Sprache eingebettet werden. Diese
  Indirektion stellt ein Problem für die Benutzung von interaktiven
  Theorembeweisern dar, da sie die Entwicklung von neuen Werkzeugen, Taktiken
  und Logiken für diese ``innere'' Sprache bedingt.

  Die vorliegende Arbeit stellt einen Compiler von der imperativen
  Programmiersprache Rust in die pur funktionale Sprache des Theorembeweisers
  Lean vor, der nicht nur generell das erste Werkzeug zur Verifikation von Rust-Programmen
  darstellt, sondern diese insbesondere auch
  mithilfe der von Lean bereitgestellten Standardwerkzeugen und -logik
  ermöglicht. Diese Transformation ist nur möglich durch spezielle Eigenschaften
  von allen validen Rust-Programmen, die die Veränderbarkeit von Werten auf
  begrenzte Geltungsbereiche einschränken und statisch durch Rusts Typsystem
  garantiert werden. Die Arbeit demonstriert den Einsatz des Compilers anhand
  der Verifikation von Realbeispielen und zeigt die Erweiterbarkeit des Projekts
  über reine Verifikation hinaus am Beispiel von asymptotischer Laufzeitanalyse auf.
\end{abstract}
\renewcommand{\abstractname}{Abstract}

\newpage

\begin{abstract}
Imperative programming, and aliasing in particular, represents a major
obstacle in formally reasoning about everyday code. By utilizing restrictions
the imperative programming language Rust imposes on mutable aliasing, we
present a scheme for shallowly embedding a substantial part of the Rust language into the
purely functional language of the Lean theorem prover. We use this scheme to
verify the correctness of real-world examples of Rust code without the need
for special semantics or logics. We furthermore
show the extensibility of our transformation by incorporating an analysis of
asymptotic runtimes.
\end{abstract}
\tableofcontents

\cleardoublepage
\pagestyle{fancy}
\fancyhf{}
\fancyhead[LE,RO]{\thepage}
\fancyhead[RE,LO]{\textit\leftmark}
\pagenumbering{arabic}

\section{Introduction}

Imperative programming languages are ubiquitous in today's software development,
making them prime targets for formal reasoning. Unfortunately, their semantics
differ from those of mathematics and logic -- the languages of formal methods -- in some
significant details, starting with the very concept of ``variables''. The problem
of mutability is only exacerbated for languages that allow references to
\emph{alias}, or point to the same memory location, enabling non-local mutation.

The standard way of verifying programs in such languages with the help of an
interactive theorem prover is to explicitly model the semantics of the language
in the language of the theorem prover, then translate
the program to this representation (a ``deep'' embedding), and finally prove the
correctness of its formalized behavior. This general approach is very flexible
and allows for the verification of meta programs such as program transformations.
The downside of the approach is that the theorem prover's tools and tactics may
not be directly applicable to the embedded language, defeating many amenities of
modern theorem provers.
Alternatively, programs can be ``shallowly'' embedded by directly translating them
into terms in the theorem prover's own language without the use of an explicit inner semantics.
This simplifies many semantic details such as the identification and
substitution of bound variables, but it is harder to accomplish the more the semantics
of the source language differs from that of the theorem prover.
Finally, if a function's specification is given directly in the code using some
sort of annotation, a \emph{verification condition generator} can directly
export the proof obligation as a formula instead of as another program. However,
inserting and maintaining the annotations for a preexisting project may be
troublesome and the specification language is often severely limited because the
verification condition generator has to parse and analyze specifications.

Regardless of the type of embedding,
an explicit heap that references can point into must generally be modeled and
passed around in order to deal with the aliasing problem. References in this model may be as simple as
indices into a uniform heap, but various logics such as separation logic~\cite{reynolds2002separation} have been developed to work on a more abstract representation and to express
aliasing-free sets of references.

Languages with more restricted forms of aliasing do exist, however.
Rust~\cite{matsakis2014rust}, a new, imperative systems programming language,
imposes on mutable references the restriction of never being aliased by any
other reference, mutable or immutable. This restriction eliminates the
possibility of data races and other common bugs created by the presence of
mutable sharing such as iterator invalidation. It furthermore enables a memory-safe
version of manual memory management and more aggressive optimizations.

While the full Rust language also provides raw pointers, which are not bound by
the aliasing restriction, and other ``unsafe'' operations, a
memory model for Rust (informal or formal) has yet to be proposed. We therefore focus on the ``safe''
subset of Rust that has no unsolved semantic details.

We utilize safe Rust's aliasing restriction to design a monadic shallow embedding of a
substantial subset of Rust
into the purely functional language of the Lean~\cite{de2015lean} theorem prover, without the need
for any heap-like indirection. This allows us to
reason about unannotated, real-world Rust code in mostly the same manner one would
reason about native Lean definitions. The monadic approach gives us further
flexibility in modeling additional effects such as function runtime.

We first discuss the simpler cases of the
translation, notably excluding mutable references, in \autoref{sec:trans}. We
show their application by giving a formal verification of Rust's
\rust{[T]::binary_search} method in \autoref{sec:binary_search}.
\autoref{sec:mutref} discusses the translation of most usages of mutable
references, which is used in \autoref{sec:fixedbitset} for a partial verification of the
\texttt{FixedBitSet} data structure. We develop a Lean library for asymptotic analysis in
\autoref{sec:asymptotic} and use it to verify the asymptotic runtime of
\rust{[T]::binary_search}. Lastly, we present some
empirical data about our coverage of the Rust language via its standard library in \autoref{sec:eval}.
\newpage
\section{Related Work}

While this thesis presents the first verification tool for Rust programs, tools
for many other imperative languages have been developed before.

The Why3 project~\cite{bobot2011why3} is notable for its generality. It provides
an imperative ML-like language \emph{WhyML} together with a verification
condition generator that can interface with a multitude of both automatic and
interactive theorem provers. While WhyML supports advanced language features such
as type polymorphism and exceptions, it does not support higher-order functions,
which are ubiquitous in Rust code.
%Specification annotations of WhyML programs are written in the logic language \emph{Why3} and thus cannot 
WhyML provides a reference type \texttt{ref} that can point to a fresh cell on
the heap and is statically checked not to alias with other \texttt{ref}
instances, but cannot point into some existing datum like Rust references can.
For example, the first of the following two WhyML functions fails to type check
because the array elements are not known to be alias-free, while the second one
will return a reference to a \emph{copy} of \verb!a[i]!.

\begin{minted}{sml}
let get_mut (a : array (ref int)) (i : int) : ref int = a[i]
let get_mut (a : array int) (i : int) : ref int = ref a[i]
\end{minted}

WhyML is also being used as an intermediate language for the verification of
programs in Ada~\cite{guitton2011hi}, C~\cite{cuoq2012frama} and Java~\cite{filliatre2007krakatoa}.
For the latter two languages, aliasing is reintroduced by way of an explicit heap.

The remarkable SeL4 project~\cite{klein2009sel4} delivers a full formal verification of an operating
system microkernel by way of multiple levels of program verification and
refinement steps. The C code that produces the final kernel binary is verified
by embedding it into the theorem prover
Isabelle/HOL~\cite{nipkow2002isabelle}, using a deep embedding for statements
and a shallow one for expressions. The C memory model used allows type-unsafe
operations by use of a byte-size heap, but as with Why3, higher-order functions are
not supported. The AutoCorres~\cite{greenaway2012bridging, greenaway2014don}
tool then transforms this representation into a shallow monadic embedding,
dealing with the `uninteresting complexities of C'~\cite{greenaway2014don} on the
way. The result is an abstracted representation that is quite similar to ours
(and in fact inspired it in part, as we shall note below), but doesn't go the
last mile of completely eliminating the heap where possible. Without explicit
no-alias annotations, the semantics of C would allow that in far fewer places than those
of Rust in any case.

It should be noted that our project, like most verification projects based on
either embedding or code extraction, relies on both
an unverified compiler and an unverified embedding tool, effectively making both
part of the trusted computing base. SeL4 is a notable exception in this,
providing (at lower optimization levels) a direct equivalence proof~\cite{sewell2013translation} between the
produced kernel binary and the verified embedded code, thus completely removing
the original C code from the trusted computing base.

\newpage
\section{Background}
We start by giving a basic introduction to our source and target
languages, focusing on the parts relevant to our work. We will discuss finer
semantic details where needed in \autoref{sec:trans} and \autoref{sec:mutref}.

\subsection{Rust}

Rust~\cite{matsakis2014rust} is a modern, multi-paradigm systems programming language sponsored
by Mozilla Research and developed as an open-source community effort. Rust is still a quite young language, with its first stable
version having been released on May 15, 2015. The two biggest Rust project as of
this writing are the Servo\footnote{\url{https://github.com/servo/servo}}~\cite{anderson2016engineering} web browser engine and
the Rust compiler \texttt{rustc}\footnote{\url{https://github.com/rust-lang/rust}} itself.

As a partly functional language, Rust is primarily inspired by ML and shares much of
its syntax, as evidenced in \autoref{fig:rustml}. However, the syntax also shows
influences by C, the dominant systems programming language of the present.
Finally, Rust also features a \emph{trait} system modeled after Haskell's type classes.

\begin{figure}[bt]
  \inputminted{rust}{code/rustml.rs}
  
  \caption{A first example of functional programming in Rust, showing algebraic
    data types, polymorphic and higher-order functions, pattern matching, type
    inference and the expression-oriented syntax}
  \label{fig:rustml}
\end{figure}

Many features of Rust other than the syntax can be explained by Rust's desire to
feature an ML-like abstraction level while still running as efficiently as C,
even on resource-constrained systems that may not allow dynamic allocation at all.
Most prominently, Rust uses manual memory management just like C and C++, but
guarantees memory safety through its \emph{ownership} and
\emph{borrowing} systems. Rust also features an \emph{unsafe} language subset that allows
everything-goes programming on the level of C, but which is usually reserved for
implementing low-level primitives on which the \emph{safe} part of the language can
then build. In general, safe Rust is (thought to be) a type-safe
language like ML and unlike either C or C++. We focus on safe Rust in the
following and in our work in order to peruse these guarantees.

Ownership describes the application of \emph{linear types} to memory management
as proposed by Wadler~\cite{wadler1990linear}. The owner of a Rust object is the binding that is responsible for freeing the
object's resources (by calling a method of the \texttt{Drop} trait), which
generally happens at the end of the binding's scope.
Because an object managing resources should only ever have one owner, types that implement
\texttt{Drop} are linear types: A value may only be used once, at which point it is
consumed and ownership is transferred to its new binding\footnote{Technically,
  because leaking resources (\ie not consuming the object at all) is a safe operation in Rust, such types are merely
  \emph{affine}. However, the distinction is not relevant for our purposes.
}. In the following example, we extract an element from a \texttt{Vec} (a dynamically-sized array
type that has to free heap space in its \texttt{Drop} implementation), after which we are not permitted to use the \texttt{Vec} again.

\begin{minted}{rust}
fn get<T>(v: Vec<T>, idx: usize) -> T {
    v[idx]
    // v will be freed here
}

let v: Vec<u32> = vec![1];
let x = get(v, 0);
// get(v, 1); // error[E0382]: use of moved value: `v`
\end{minted}

One way to retain access to the \texttt{Vec} would be to also return it from the
function, regaining ownership. However, since \texttt{T} in general is an linear
type too, \texttt{get} would have to remove the indexed element before returning
the \texttt{Vec}.

A much better alternative is to use \emph{references}, which provide standard
pointer indirection. Because a reference does not take ownership of the pointee,
creating it is also called \emph{borrowing}.

\begin{minted}{rust}
fn get<T>(v: &Vec<T>, idx: usize) -> &T {
    &v[idx]
}

let v: Vec<u32> = vec![1];
let x = get(&v, 0); // x: &u32
\end{minted}

Here \rust{&T} represents an immutable reference to a value of type \rust{T}. Note that the compiler would stop us if we tried to return \texttt{v[idx]} by value:

\begin{verbatim}
error[E0507]: cannot move out of indexed content
\end{verbatim}

Still, coming from other languages with manual memory management, this might
look like a potentially unsafe thing to do: The function signature does not tell
the callee that the returned reference is only valid as long as the
\texttt{Vec}. Even Wadler tells us that a temporary reference to a linear value
must be checked not to escape from the local scope. Indeed, it seems like the following program should produce a dangling pointer.

\begin{minted}{rust}
fn dangling() -> u32 {
    let x = {
        let v: Vec<u32> = vec![1];
        get(&v, 0)
        // v will be freed here
    };
    *x
}
\end{minted}

However, the Rust compiler will stop us from doing this, printing an
elaborate error message:

\begin{verbatim}
error: `v` does not live long enough
|
|         get(&v, 0)
|              ^ does not live long enough
|     };
|     - borrowed value only lives until here
|     *x
| }
| - borrowed value needs to live until here
\end{verbatim}

The compiler must have had some information about the relationship of \texttt{x}
and \texttt{v} in order to deduce this without resorting to inter-procedural
analysis. It turns out that the full signature of the \texttt{get} function is as follows:

\begin{minted}{rust}
fn get<'a, T>(v: &'a Vec<T>, idx: usize) -> &'a T
\end{minted}

\rust!'a! is called a \emph{formal lifetime parameter}. It
specifies that the returned reference is indeed only valid as long as the first
argument. By integrating lifetimes into the type system like this, Rust can
reason about references even when confronted with complex, inter-procedural, higher-order reference lifetime relations.

While we have solved the dangling pointer problem for immutable data, mutability
as so often aggravates the problem.

\begin{minted}{rust}
fn dangling2() -> u32 {
    let mut v: Vec<u32> = vec![1];
    let x = get(&v, 0);
    // remove all elements from v
    v.clear(); // shorthand for (&mut v).clear();
    *x
    // v will be freed here
}
\end{minted}

By clearing the vector while we still hold a reference to its content, we should
again produce a dangling pointer -- even though this time, \rust{v} indeed
outlives \rust{x}. Fortunately, the Rust compiler will again stop us:

\begin{minted}{text}
error[E0502]: cannot borrow `v` as mutable because it is also borrowed as immutable
|
|     let x = get(&v, 0);
|                  - immutable borrow occurs here
|     // remove all elements from v
|     v.clear(); // shorthand for (&mut v).clear();
|     ^ mutable borrow occurs here
|     *x
| }
| - immutable borrow ends here
\end{minted}

We have finally arrived at the aliasing problem: In a language with manual
memory management, we can create type unsafety through the mere existence of two
pointers, at least one of them mutable, to the same datum. Thus, Rust detects
and forbids any occurrences of mutable aliasing, as shown above.

%The beauty of Rust's solution to safe managed memory management is that the absence of mutable aliasing solves
The beauty of forbidding mutable aliasing is that it solves many sources of bugs
in imperative programs even outside of managed memory management. Indeed, as
Wadler notes, it makes mutable references safe even in a referentially
transparent language: ``In order for destructive updating of a value to be safe,
it is essential that there be only one reference to the value when the update
occurs''~\cite{wadler1990linear}. While Rust does introduce APIs such as for I/O that break referential
transparency, the absence of mutable aliasing still provides safety guarantees
that are usually only attributed to purely functional languages, first and
foremost among them the elimination of data races. By focusing on a subset of
Rust and its APIs that is truly referentially transparent, we obtain a
sufficiently narrow gap between Rust and the purely functional language Lean
that our transformation between them becomes feasible.

\subsection{Lean}
\newpage

\section{Basic Transformation}
\label{sec:trans}

In this section, we describe the basic translation from Rust to Lean that
includes pure code as well as mutable local variables and loops, but not mutable
references (see~\autoref{sec:mutref}). We roughly follow the structure of the
Rust Reference\footnote{\url{https://doc.rust-lang.org/reference.html}}.

\subsection{The MIR}

\begin{figure}[!bp]
  \centering
  \begin{tikzpicture}
    \node (1) [block] { source };
    \node (2) [block,below=of 1] { AST };
    \node (3) [block,below=of 2] { HIR };
    \node (4) [block,below=of 3] { MIR };
    \node (5) [block,below=of 4] { LLVM IR };
    \node (6) [block,right=3cm of 4] { Lean };
    \draw[->] (1) to (2);
    \draw[->] (2) edge[loop left] node[align=right] {macro expansion\\name resolution} (2);
    \draw[->] (2) to (3);
    \draw[->] (3) edge[loop left] node[align=right] {lifetime resolution\\validity checks} (3);
    \draw[->] (3) to (4);
    \draw[->] (4) edge[loop left] node[align=right] {optimizations\\borrow checking} (4);
    \draw[->] (4) to (5);
    \draw[->] (4) to node {our work} (6);
  \end{tikzpicture}
  \caption{Overview of the Rust compiler pipeline and our work in that context}
  \label{fig:mir}
\end{figure}

Because Rust makes extensive use of inference algorithms for types, lifetimes and typeclasses,
correctly parsing Rust code is no small feat. Therefore, we use the Rust
Compiler \texttt{rustc} itself as a frontend and work on the completely explicit and
much simpler \emph{mid-level intermediate representation} (MIR)
(\autoref{fig:mir}). By writing our translation program in Rust, we can import
the \texttt{rustc} libraries to gain access to the MIR and many convenient
helper functions.

The MIR is a control flow graph (CFG) representation where a basic block consists
of a list of statements followed by a terminator that (conditionally or
unconditionally) transfers control to other basic blocks. For readability,
this section will mostly argue on the Rust source level, but the graph structure
will be important for translating control flow.

%Apart from annotation statements and ones that will be inserted in the backend, the only statement kind we have to handle is the assignment of an rvalue to an lvalue. 

\subsection{Programs and files}

Rust's unit of compilation is called a \emph{crate}. A crate consists of one or
more \texttt{.rs} files and can be compiled to an executable or library. Files
inside a crate may freely reference declarations between them. On the other
hand, Lean files may only import other files non-recursively and declarations
must be strictly sorted in order of usage for termination checking. We therefore
translate a crate into a single Lean file and perform a topological sort on its
declarations. While Lean does support explicit declarations of mutually
recursive types and functions, we have not yet encountered such declarations in
Rust code as part of our formalization work and thus have not implemented support for them so far.

In detail, our tool creates a file called \verb!generated.lean! in a separate
folder for each crate and connects them using Lean's \lean{import} directive
according to the inter-crate dependencies. The user can additionally create a
\verb!pre.lean! file that will automatically be imported and can be used for
axiomatizations as well as a \verb!config.toml! file that can influence the
translation -- see below for examples. We use a third Lean file \verb!thy.lean! per crate
for the proofs, which will import both the generated code and proof files from
other crates.

\subsection{Types}

\subsubsection{Primitive Types}

Rust's primitive types are the boolean type, machine-independent and machine-dependent integer
types, tuples, arrays, slices, and function types.

Following AutoCorres's design, we map the primitive integer types to
Lean's native arbitrary-sized types and instead handle overflow explicitly
during computation (\autoref{sec:arith}).

\begin{minted}{lean}
abbreviation u8 [parsing_only] := nat
abbreviation u16 [parsing_only] := nat
abbreviation u32 [parsing_only] := nat
abbreviation u64 [parsing_only] := nat
abbreviation usize [parsing_only] := nat

abbreviation i8 [parsing_only] := int
// ...

definition u8.bits [reducible] : ℕ := 8
// ...

definition usize.bits : ℕ := 16
lemma usize.bits_ge_16 : usize.bits ≥ 16 := dec_trivial
attribute usize.bits [irreducible]
\end{minted}

For the machine-size integer types \rust{usize} and \rust{isize}, we only expose
the conservative assumption that their bit sizes are at least 16. We still define
\rust{usize.bits} to be exactly 16 so that it is computable, but by then marking
the definition as \rust{[irreducible]}, this fact is only accessible in proofs
when explicitly unfolding the definition.
When a proof does rely on the bounds of a parameter, we can add a separate
hypothesis, for which we make use of typeclasses. The bounds of an expression
can often be determined just from partial information, such as with unsigned division.

\begin{minted}{lean}
definition is_bounded_nat [class] (bits x : ℕ) := x < 2^bits
abbreviation is_usize := is_bounded_nat usize.bits

lemma div_is_bounded_nat [instance] (bits x y : ℕ)
  [is_bounded_nat bits x] : is_bounded_nat bits (x / y) := ...
\end{minted}

We use the same approach for arrays (\rust{[T; N]}) and slices (\rust{&[T]}),
mapping both the to arbitrary-length \lean{list} type. While Rust arrays have a
constant length encoded in the type, slices are dynamic views into contiguous sequences
like arrays or \rust{Vec}s and bounded only by the memory size. The latter fact is important
when trying to prove that a \rust{usize} counter can reach all indices in a slice.

\begin{minted}{lean}
abbreviation array [parsing_only] (A : Type₁) (n : ℕ) := list A
abbreviation slice [parsing_only] := list

definition is_slice [class] {T : Type₁} (xs : slice T) :=
length xs ≤ 2^usize.bits
\end{minted}

\subsubsection{Structs and enums}

Because Rust does not feature inheritance, struct types and enumerated types are
true Algebraic Data Types and can directly be translated to their Lean
equivalents (\lean{structure} and \lean{inductive}, respectively).

\subsubsection{References}

An immutable reference \rust{&'a T} is checked by the Rust compiler not to alias
with any mutable reference and thus can be safely replaced with the translation
of \rust{T} itself. We drop all lifetime specifiers in general because we trust
the Rust compiler to already have made the memory safety checks.

We will discuss mutable references in \autoref{sec:mutref}.

\subsection{The semantics monad}

The core part for representing Rust's dynamic semantics is the monadic embedding. While
higher-order unification issues in the current Lean version prevent us from
outright parameterizing the embedding by an arbitrary monad instance, we still
try to keep the interface of our specific monad abstract so that the monad can be
extended in the future.

We currently model abnormal termination\footnote{unspecified behavior like integer
overflow and \emph{panics} from out-of-bounds array accesses or explicit \rust{panic!}
calls. Rust does not have exceptions.} and
nontermination as well as an abstract step counter for asymptotic run time analysis.

\begin{minted}{lean}
definition sem (A : Type₁) := option (A × ℕ)
\end{minted}

We provide the standard monadic operations on the type, including a
\texttt{do}-notation. The model-specific operations are \lean{mzero}
indicating abnormal termination/nontermination, and \lean{sem.incr}, which
increments the step counter (if any). An increment of one is emitted around
every Rust function call and before each loop iteration.

\begin{minted}{lean}
definition mzero {A : Type₁} : sem A := none
definition return {A : Type₁} (x : A) : sem A := some (x, 0)

definition sem.incr {A : Type₁} (n : ℕ) : sem A → sem A
| (some (x, k)) := some (x, k+n)
| none          := none

definition sem.bind {A B : Type₁} (m : sem A) (f : A → sem B)
  : sem B :=
option.bind m (λs, match s with
| (x, k) := sem.incr k (f x)
end)

infixl ` >>= `:2 := sem.bind
\end{minted}

The semantics monad follows the usual monad laws, which we will make use of in proofs.

\begin{minted}{lean}
lemma return_bind {A B : Type₁} {a : A} {f : A → sem B}
  : (return a >>= f) = f a := ...
lemma bind_return {A : Type₁} {m : sem A} : (m >>= return) = m := ...
lemma bind.assoc {A B C : Type₁} {m : sem A} {f : A → sem B}
  {g : B → sem C} : (m >>= f >>= g) = (m >>= (λx, f x >>= g)) := ...
\end{minted}

\subsection{Statements and control flow}

The local state of a Rust function consists of its arguments, variables, and
temporaries (variables introduced by the compiler). Without mutable references,
these locals can only be manipulated by assignments, the single statement kind
available in the MIR. In linear code, keeping track of assignments is as easy as
transforming them to redeclarations.

\vspace{1em}\noindent\begin{minipage}{0.4\textwidth}
  \begin{minted}{rust}
p.x += 1;
  \end{minted}
\end{minipage}
\begin{minipage}{0.6\textwidth}
  \begin{minted}{lean}
let p = Point { x = p.x + 1, ..p };
  \end{minted}
\end{minipage}\vspace{1em}

Nonlinear control flow is introduced by Rust's \rust{if} and \rust{match}
constructs as well as its three loop constructs (which have a single common
representation in the MIR). We map the first two cases to Lean's corresponding
constructs of the same names.

\vspace{1em}\noindent\begin{minipage}{0.4\textwidth}
  \begin{minted}{rust}
let x = if b {1} else {0};
x & 1
  \end{minted}
\end{minipage}
\begin{minipage}{0.33\textwidth}
\begin{tikzpicture}
  \node (1) [block] { \rust{if b} };
  \node (2) [block,below=of 1,xshift=-9mm] { \rust{x = 0;} };
  \draw[->] (1) to node[left] {\rust{false}} (2);
  \node (3) [block,below=of 1,xshift=9mm] { \rust{x = 1;} };
  \draw[->] (1) to node[right] {\rust{true}} (3);
  \node (4) [block,below=of 2,xshift=9mm] { \rust{ret = x & 1; return} };
  \draw[->] (2) to (4);
  \draw[->] (3) to (4);
\end{tikzpicture}
\end{minipage}
\begin{minipage}{0.3\textwidth}
  \begin{minted}{lean}
if b = bool.tt then
  let x := 1 in
  x & 1
else
  let x := 0 in
  x & 1
  \end{minted}
\end{minipage}\vspace{1em}

As can be seen, we currently translate each branch of a conditional block
terminator independently, which can lead to code duplication if those branches
converge again. While this has not manifested any problems in our verification
work so far, we may want to mitigate it in the future by factoring out the
common translated code into a separate definition.

\begin{figure}[!b]
\hspace{1cm}\begin{minipage}{0.4\textwidth}
  \begin{minted}{rust}
fn f() {
  let mut x = 0;
  while x < 10 {
      x += 1;
  }
}
  \end{minted}
\end{minipage}
\begin{minipage}{0.4\textwidth}
  \newsavebox{\mintedbox}
  \begin{lrbox}{\mintedbox}
    \begin{minipage}{1.8cm}
\begin{minted}{rust}
x = 0;
if x < 10
\end{minted}
    \end{minipage}
  \end{lrbox}
\begin{tikzpicture}
  \node (start) {};
  \node (1) [block,label=left:1,below=5mm of start] {\usebox{\mintedbox}};
  \draw[->] (start) to (1);
  \node (2) [block,label=left:2,below=of 1,xshift=-15mm] { \rust{return} };
  \draw[->] (1) to node[left] {\rust{false}} (2);
  \node (3) [block,label=left:3,below=of 1,xshift=15mm] { \rust{x = x + 1;} };
  \draw[->] (1) to node[right] {\rust{true}} (3);
  \draw[->] (3) to[bend left=45] (1);
\end{tikzpicture}
\end{minipage}

\caption{A \rust{while} loop and the corresponding (simplified) MIR graph.
  Blocks 1 and 3 from a strongly connected component, which is dominated by
  block 1, the loop header.}
\label{fig:scc}

\end{figure}

We do need to factor out common code in the case of loops. There is no special
terminator signifying loops in the MIR; instead, we have to search for
(nontrivial) strongly connected components (SCCs) of basic blocks (\autoref{fig:scc}). Because Rust's
control flow is \emph{reducible} (notably, lacking a \emph{goto} instruction),
we may assume that such an SCC can only be entered from a single node
(\emph{dominating} the SCC). With this, we can describe the semantics of the SCC
in more traditional terms of iteration: The dominating node is the \emph{loop
  header}, while the rest of the SCC is the \emph{body}. Jumping back to the
header signifies a new iteration, while jumping out of the SCC means breaking
the loop. By breaking up the SCC at the header, we can thus translate a single
iteration to a function of type

\begin{minted}{lean}
State → sem (State + Res)
\end{minted}

\noindent{}that takes a tuple \lean{State} of loop variables and either returns the new
state for the next iteration, or a value of the source function's return type
\lean{Res} when breaking out of the loop. We tie this into a single value of
type \lean{sem Res} by use of a general \emph{loop combinator}.

\subsubsection{The loop combinator}

The loop combinator has the signature

\begin{minted}{lean}
noncomputable definition loop {State Res : Type₁}
  (body : State → sem (State + Res)) (s : State) : sem Res
\end{minted}
Its task is to apply \rust{body} repeatedly (starting with \rust{s}) until some
\rust{Res} is returned; if the loop does not terminate, it returns \rust{mzero}
(which \rust{body} may also return by itself). Termination for arbitrary values
of \rust{body} obviously is not a decidable property. Therefore we will have to leave
the constructive subset of Lean, as signified by the \lean{noncomputable}
specifier. The (simplified) translation of the Rust code in \autoref{fig:scc} via
\lean{loop} is as follows:

\begin{minted}{lean}
definition f.loop_1 (x : i32) : sem (i32 + unit) :=
if x < 10 then
  let x := x + 1 in
  return (sum.inl x)
else
  return (sum.inr unit.star)

definition f : sem unit :=
let x := 0 in
loop f.loop_1 x
\end{minted}

As a total, purely functional language, Lean cannot express iteration directly,
and the only primitive kind of recursion available in Lean is structural recursion
over an inductive datatype. On top of structural recursion, the Lean standard
library defines the more general concept of \emph{well-founded} recursion: A
relation \lean{R : A → A → Prop} on a type \lean{A} is well-founded if every
element of \lean{A} is \emph{accessible} through the relation, which is defined
inductively as all predecessors of the element under the relation being accessible.

\begin{minted}{lean}
inductive acc {A : Type} (R : A → A → Prop) : A → Prop :=
intro : ∀x, (∀ y, R y x → acc R y) → acc R x

inductive well_founded [class] {A : Type} (R : A → A → Prop) : Prop :=
intro : (∀ a, acc R a) → well_founded R
\end{minted}

Using structural recursion over the \lean{acc} predicate, the standard library
defines a fixed-point combinator for functionals respecting a well-founded
relation, and proves that the combinator satisfies the fixpoint equation.

\begin{minted}{lean}
namespace well_founded
section
  variables {A : Type} {C : A → Type} {R : A → A → Prop}

  definition fix [well_founded R] (F : Πx, (Πy, R y x → C y) → C x)
    (x : A) : C x := ...

  theorem fix_eq [well_founded R] (F : Πx, (Πy, R y x → C y) → C x)
    (x : A) : fix F x = F x (λy h, fix F y) := ...
end
end well_founded
\end{minted}

We use well-founded recursion to define \lean{loop}: If repeatedly applying
\emph{body} to \emph{s} yields a sequence of states,
this sequence will terminate iff there exists a well-founded relation on
\lean{State} such that the sequence is a descending chain.
This is true because descending chains in well-founded relations are finite, and
conversely a finite sequence $s_1 = s, \dots, s_n$ is a descending chain in the
trivial well-founded relation $R = \{(s_{i+1}, s_i) | 1 \le i < n\}$.

In the formalization, given a well-founded relation \lean{R} on \lean{State}, we first have to take care of lifting it to a well-founded relation \lean{R'} on \lean{State + Res}.

\begin{minted}{lean}
section
  parameters {State Res : Type₁}
  parameter (body : State → sem (State + Res))
  parameter (R : State → State → Prop)

  definition State' := State + Res

  definition R' : State' → State' → Prop
  | (inl s') (inl s) := R s' s
  | _        _       := false

  private lemma R'.wf [instance] [well_founded R] : well_founded R' := ...
\end{minted}

We can then wrap \lean{body} in a functional respecting \lean{R'} that we can pass to \lean{well_founded.fix}.

\begin{minted}{lean}
  definition F (x : State') (f : Π (x' : State'), R' x' x → sem State') : sem State' :=
  match x with
  | inr _ := mzero -- unreachable
  | inl s :=
    do x' ← sem.incr 1 (body s);
    match x' with
    | inr r := return (inr r)
    | x'    := if H : R' x' x then f x' H else mzero
    end
  end

  definition loop.fix [well_founded R] (s : State) : sem Res :=
  do x ← well_founded.fix F (inl s);
  match x with
  | inr r := return r
  | inl _ := mzero -- unreachable
  end
\end{minted}

Finally, we implement \lean{loop} by choosing any well-founded relation \lean{R} that makes
the loop terminate, if any, or else return \lean{mzero}.

\begin{minted}{lean}
  definition terminating (s : State) :=
  ∃ Hwf : well_founded R, loop.fix s ≠ mzero

  noncomputable definition loop (s : State) : sem Res :=
  if Hex : ∃ R, terminating R s then
    @loop.fix (classical.some Hex) _ (classical.some (classical.some_spec Hex)) s
  else mzero
\end{minted}

Here we make use of the \emph{dependent if-then-else} notation that allows us to
test for a property and then bind a name to a proof of it in case it holds. We
then destructure that proof object to obtain the relation and its
well-foundedness proof so that we can pass them to \lean{loop.fix}. The
\lean{classical.some} and \lean{classical.some_spec} definitions are based on
Hilbert's epsilon operator.

\begin{minted}{lean}
noncomputable definition classical.some {A : Type} {P : A → Prop} (H : ∃x, P x) : A := ...
theorem classical.some_spec {A : Type} {P : A → Prop} (H : ∃x, P x) : P (some H) := ...
\end{minted}

The use of \lean{classical.some} as well as the undecidable conditional
\lean{∃ R, terminating R s} make \lean{loop} non-computable.

When verifying loops, we will first verify the corresponding application of
\lean{loop.fix} using a specific well-founded relation, for which we can prove a
convenient fixpoint equation.

\begin{minted}{lean}
  theorem loop.fix_eq
    {R : State → State → Prop} [well_founded R] {s : State} :
    loop.fix R s =
      do x' ← sem.incr 1 (body s);
      match x' with
      | inl s' := if R s' s then loop.fix R s' else mzero
      | inr r  := return r
      end := ...
\end{minted}

If the application of \lean{loop.fix} terminates, we can show that the original
application \lean{loop} will do so too with the same return value,
via a helper lemma that says that all terminating \lean{loop.fix} applications are equal.

\begin{minted}{lean}
  lemma loop.fix_eq_fix
    {R₁ R₂ : State → State → Prop} [well_founded R₁] [well_founded R₂]
    {s : State}
    (Hterm₁ : loop.fix R₁ s ≠ mzero)
    (Hterm₂ : loop.fix R₂ s ≠ mzero) :
    loop.fix R₁ s = loop.fix R₂ s := ...

  theorem loop.fix_eq_loop
    {R : State → State → Prop} [well_founded R]
    {s : State}
    (Hterm : loop.fix R s ≠ mzero) :
    loop.fix R s = loop s := ...
\end{minted}

\subsection{Expressions}

\subsubsection{Arithmetic expressions}
\label{sec:arith}
\newpage
\section{Case Study: Verification of \texttt{[T]::binary\_search}}
\label{sec:binary_search}

As a first test of the translation tool, we set out to verify the correctness
of the binary search implementation in the Rust standard library, an algorithm of
medium complexity.

\subsection{The Rust Implementation}

Before we can even tackle the algorithmic complexity, we have to cope with the
design complexity of a real-world library. The public implementation of the
\rust{binary_search} method implemented on any slice type can be found in the \rust{collections} crate.

\begin{minted}{rust}
use core::slice as core_slice;

impl<T> [T] {
  ...

  /// Binary search a sorted slice for a given element.
  ///
  /// If the value is found then `Ok` is returned, containing the
  /// index of the matching element; if the value is not found then
  /// `Err` is returned, containing the index where a matching
  /// element could be inserted while maintaining sorted order.
  ///
  /// ...
  pub fn binary_search(&self, x: &T) -> Result<usize, usize>
    where T: Ord {
    core_slice::SliceExt::binary_search(self, x)
  }
}
\end{minted}

As we can see from the its documentation and signature, the method
is very general: It works on all slices whose element type implements the
\rust{Ord} trait, and it returns information in both the success and the failure
case. The implementation, however, turns out to be merely a redirection to a
trait method in the base crate \rust{core}. This trait has a single
implementation, for the slice type.

\begin{minted}{rust}
pub trait SliceExt {
  type Item;

  fn binary_search(&self, x: &Item) -> Result<usize, usize>
    where Item: Ord;
  fn len(&self) -> usize;
  fn is_empty(&self) -> bool { self.len() == 0 }
  ...
}

impl<T> SliceExt for [T] {
  type Item = T;

  fn binary_search(&self, x: &T) -> Result<usize, usize> where T: Ord {
    self.binary_search_by(|p| p.cmp(x))
  }
  fn len(&self) -> usize { ... }
  ...

}
\end{minted}

This indirection seems pointless at first, but follows from a technical
restriction: There may be at most one \rust{impl} block for a primitive type
like \rust{[T]}. Because the \rust{core} crate does not depend on the existence
of a heap allocator, but some methods on \rust{[T]} like its merge sort
implementation do need dynamic allocation, the \rust{impl} block is only
declared in the later \rust{collections} crate. Since \rust{binary_search} does
not need an allocator, it should still reside in \rust{core}, and instead is
associated to the slice type via the helper trait.

\begin{listing}[!bt]
\begin{minted}{rust}
fn binary_search_by<'a, F>(&'a self, mut f: F) -> Result<usize, usize>
    where F: FnMut(&'a T) -> Ordering
{
    let mut base = 0usize;
    let mut s = self;

    loop {
        let (head, tail) = s.split_at(s.len() >> 1);
        if tail.is_empty() {
            return Err(base)
        }
        match f(&tail[0]) {
            Less => {
                base += head.len() + 1;
                s = &tail[1..];
            }
            Greater => s = head,
            Equal => return Ok(base + head.len()),
        }
    }
}
\end{minted}
  
\caption{Implementation of the \rust{binary_search_by} method. A subslice
  \rust{s} of \rust{self} is iteratively bisected until it is empty or the
  element has been found. The \rust{tail[1..]} \emph{slicing syntax} is
  syntax sugar for \rust{tail.index(RangeFrom{start: 1})}.}
\label{lst:binary_search_by}
\end{listing}

This final version of \rust{binary_search}, which we represent as
\rust{core::<[T] as SliceExt>::binary_search}, is implemented by way of a more
general method \rust{binary_search_by} that takes a comparison function instead
of being constrained to \rust{Ord} (\autoref{lst:binary_search_by}). This
method, finally, turns out to be much more abstract than one might expect: Instead of
the standard binary search implementation that iteratively reduces the search range via two
indices, the range is represented as a subslice and manipulated via high-level
slice methods such as \rust{split_at}. The reasoning behind this is a great show
case for Rust's zero-cost (or even negative-cost, in this case) abstractions
philosophy -- the abstract implementation actually surpasses a direct
implementation in terms of efficiency because it helps the compiler to eliminate
all bounds checks in it. It also elegantly avoids the common
pitfall\footnote{https://research.googleblog.com/2006/06/extra-extra-read-all-about-it-nearly.html}
of a potential integer overflow in less abstract code like \rust{mid = (low + high) / 2}.

\begin{sidewaysfigure}
  \includegraphics[width=\textheight]{deps}
  \caption[A complete graph of the dependencies of \rust{binary_search}]{A
    complete graph of the translation dependencies of \rust{binary_search} in
    the \rust{core} crate,
    distinguishing between \tikz[baseline=-0.3em]\node[draw, shape=ellipse] {functions};,
    \tikz[baseline=-0.3em]\node[draw, shape=rectangle] {types};,
    \tikz[baseline=-0.3em]\node[draw, shape=diamond, aspect=2] {traits};,
    and \tikz[baseline=-0.3em]\node[draw, shape=trapezium] {trait
      implementations};. Axiomatized items that use unsafe code in the original
    implementation are marked by dashed borders. Because
    we eagerly resolve trait method calls where possible, such as to the
    \rust{index} method of \rust{Index<RangeFrom<usize>>} for \rust{[T]}, we can
    avoid some dependencies like the full \rust{Index} implementation for
    \rust{[T]}, and even the trait itself.
  }
  \label{fig:deps}
\end{sidewaysfigure}

For our purposes, the abstract implementation primarily means a fair number of
additional dependencies we have to support and inspect~(\autoref{fig:deps}). All
in all, \rust{binary_search} turned out to be an ideal first test not only because
of its algorithmic complexity, but also because of its use of numerous Rust
language features including enums, structs, traits with associated types and
default methods, higher-order functions, and loops.

\subsection{Prelude: Coping with Unsafe Dependencies}

When trying to translate the \rust{binary_search} method including its
dependencies, we will not get back a working definition at first. Our tool
refuses to translate some dependencies because they use unsafe code, as marked
in \autoref{fig:deps}. We will have to translate these functions manually,
basically adding the correctness of their translation as axioms to the project.

Apart from our custom translation of \rust{FnMut} we discussed in
\autoref{sec:lambda}, both axiomatized functions operate on slices and are straightforward to
implement using our identification of slices with Lean lists.

\begin{minted}{lean}
-- Returns the number of elements in the slice.
definition «[T] as core.slice.SliceExt».len {T : Type₁} (self : slice T) : sem nat :=
return (list.length self)

-- Implements slicing with syntax `&self[begin .. end]`.
-- Returns a slice of self for the index range [`begin`..`end`).
-- This operation is `O(1)`.
-- Requires that `begin <= end` and `end <= self.len()`,
-- otherwise slicing will panic.
definition «[T] as core.ops.Index<core.ops.Range<usize>>».index {T : Type₁} (self : slice T) (index : Range usize) : sem (slice T) :=
sem.guard (Range.start index ≤ Range.«end» index ∧ Range.«end» index ≤ list.length self)
  (return (list.firstn (Range.«end» index - Range.start index) (list.dropn (Range.start index) self)))
\end{minted}

The latter method presents a
small technical hurdle: It is dependent on other translation products,
specifically the \lean{Range} structure. Instead of having to axiomatize that
perfectly translatable item and adding both definitions manually to the
\verb!pre.lean! file, we instruct the translator in the \verb!config.toml! file to inject our
Lean definition of \rust{index} as the translation of the Rust definition on-the-fly.

\begin{minted}{text}
[replace]
"«[T] as core.ops.Index<core.ops.Range<usize>>».index" = "..."
\end{minted}

\subsection{Formal Specification}

Going back to the original definition of \rust{[T]::binary_search}, we translate
the documented behavior into a Lean predicate.

\begin{minted}{lean}
parameters {T : Type₁} [Ord T]
parameter self : slice T
parameter needle : T -- a more descriptive name for the parameter `x`

inductive binary_search_res : Result usize usize → Prop :=
| found     : Πi, list.nth self i = some needle → binary_search_res (Result.Ok i)
| not_found : Πi, needle ∉ self → sorted (list.insert_at self i needle) →
  binary_search_res (Result.Err i)
\end{minted}

It is specifications like these where the power of shallow embeddings really
shines: We can freely mix and match Rust types and standard Lean functions and
constructs. In fact, we will have to do some more mixing of these two worlds to make
the definition valid: While we have copied the assumption \rust{T : Ord} from
the \rust{binary_search} method, the \lean{sorted} predicate expects T to
implement Lean's own ordering typeclass. We therefore introduce a new typeclass
\lean{Ord'} that merges both typeclasses -- or rather, in the Lean case, the
subclass of decidable, linear orders.

\begin{minted}{lean}
definition ordering {T : Type} [decidable_linear_order T] (x y : T) : cmp.Ordering :=
if x < y then Ordering.Less
else if x = y then Ordering.Equal
else Ordering.Greater

structure Ord' [class] (T : Type₁) extends Ord T, decidable_linear_order T :=
(cmp_eq : ∀ x y : T, Σ k, cmp x y = some (ordering x y, k))
\end{minted}

After changing the \lean{parameter} definition to \lean{Ord' T}, the
specification typechecks. We need two more (sensible) hypotheses before we can
prove that \lean{binary_search} upholds the specification.

\begin{minted}{lean}
hypothesis Hsorted : sorted self
hypothesis His_slice : is_slice self

...

theorem binary_search.spec : sem.terminates_with
  binary_search_res
  (binary_search self needle) := ...
\end{minted}

\subsection{Proof}

The full correctness proof is about 170 lines in Lean's tactic mode. We will not
discuss the individual steps or the Lean tactic syntax here, but focus on the main
proof steps.

After unfolding the \lean{binary_search} and \lean{binary_search_by}
definitions and some simplifications, we quickly reduce the proof obligation
down to the central loop.

\begin{minted}{lean}
⊢ sem.terminates_with binary_search_res
    (loop loop_4 (closure_5594.mk needle, 0, self))
\end{minted}

Here \lean{loop_4} is the loop body extracted from \lean{binary_search_by},
which is passed to the loop combinator \lean{loop} together with the initial
loop state. The loop state is the triple \lean{(f, base, s)} of local variables
mutated in the loop, initialized to the closure from \lean{binary_search}
(capturing \lean{needle}), \lean{0}, and \lean{self}, respectively. As described
in \autoref{sec:loop}, we can reduce the goal to one basing the loop on a
specific relation by use of the lemma \lean{loop.fix_eq_loop}.

\begin{minted}{lean}
abbreviation f₀ := closure_5594.mk needle
abbreviation loop_4.state := closure_5594 T × usize × slice T
definition R := measure (λ st : loop_4.state, length st.2)
...

⊢ sem.terminates_with binary_search_res
  (loop.fix loop_4 R (f₀, 0, self))
\end{minted}

\lean{measure} lets us create a well-founded relation on the loop state triple
by comparing the length of \lean{s}. We will not be able to show the new goal
directly via well-founded induction over \lean{R}, instead we first have to
generalize it. For that we first declare the loop invariants (which we obtained by
the non-sophisticated method of repeated try-and-error).

\begin{minted}{lean}
variables (base : usize) (s : slice T)

structure loop_4_invar :=
(s_in_self  : s ⊑ₚ (dropn base self))
(insert_pos : sorted.insert_pos self needle ∈ '[base, base + length s])
(needle_mem : needle ∈ self → needle ∈ s)
\end{minted}

These say that
\begin{enumerate}
\item \lean{s} is a contiguous subsequence of the original slice \lean{self} starting at \lean{base}; here \lean{⊑ₚ} is a
  notation for the (non-strict) list prefix order that will come in handy at multiple points
  in the proof.
\item inserting \lean{needle} at the first position in \lean{self} that will
  keep it sorted will insert it inside or adjacent to \lean{s}.
\item if \lean{needle} is at all in the original slice, it will also be in
  \lean{s}. If this is the case, this invariant will imply the previous one, but in
  general they are independent.
\end{enumerate}

Because the invariants trivially hold for the initial state, we can generalize
the goal.

\begin{minted}{lean}
⊢ loop_4_invar base s → sem.terminates_with binary_search_res
    (loop.fix loop_4 R (f₀, base, s))
\end{minted}

There is no need to generalize \rust{f₀} because we know it is a non-modifying closure
and thus the variable \rust{f} will always contain that value.

After applying well-founded recursion, we unroll one iteration of
\lean{loop.fix} via the lemma \lean{loop.fix_eq} from \autoref{sec:loop} and
apply the induction hypothesis on the loop remainder to reduce the goal to that
single iteration.

\begin{minted}{lean}
inductive loop_4_step : loop_4.state → Prop :=
mk : Π base' s', loop_4_invar base' s' → length s' ≤ length s / 2 → length s ≠ 0 →
  loop_4_step (f₀, base', s')


⊢ loop_4_invar base s → sem.terminates_with
    (sum.rec (loop_4_step s) binary_search_res)
    (loop_4 R (f₀, base, s))
\end{minted}

If the iteration breaks the loop (returns some \lean{sum.inr}), we need the
result to fulfill the top-level specification \lean{binary_search_res}.
Otherwise, if the loop produces
some new loop state \lean{(f₀, base', s')}, the loop invariants should be upheld
together with a loop \emph{variant} saying that the length of \lean{s} has at
least halved. Together with the information that \lean{length s ≠ 0}, this
implies \lean{length s' < length s} and ensures we can apply the induction
hypothesis. We will
need the former two stronger statements for proving the function's logarithmic
complexity below.

The remainder of the proof, while tedious, uses mostly basic reasoning. We split
the goal according to the \rust{if} and \rust{match} branches in the original
code and, depending on the return value in each case, show that
\lean{loop_4_invar} or \lean{binary_search_res} is upheld. We prove that neither
of the two additions in the code overflows by showing that they are bounded by
\lean{list.length self}, which by the assumption \lean{is_slice self} fits into
the \rust{usize} type.

\newpage
\section{Transformation of Mutable References}
\label{sec:mutref}

\section{Case Study: Verification of \texttt{fixedbitset}}
\label{sec:fixedbitset}

\section{Conclusion and Future Work}

\newpage
\bibliographystyle{abbrv}
\bibliography{bib}

\cleardoublepage
\pagestyle{empty}
\section*{Erklärung}

  \vspace{20mm}
  Hiermit erkläre ich, Sebastian Andreas Ullrich, dass ich die vorliegende Masterarbeit selbst\-ständig
verfasst habe und keine anderen als die angegebenen Quellen und Hilfsmittel
benutzt habe, die wörtlich oder inhaltlich übernommenen Stellen als solche kenntlich gemacht und
die Satzung des KIT zur Sicherung guter wissenschaftlicher Praxis beachtet habe.
  \vspace{20mm}
  \begin{tabbing}
  \rule{4cm}{.4pt}\hspace{1cm} \= \rule{7cm}{.4pt} \\
 Ort, Datum \> Unterschrift
  \end{tabbing}

\end{document}
