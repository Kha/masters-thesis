\documentclass[12pt,a4paper,twoside]{article}


\usepackage[utf8]{inputenc}
\usepackage[T1]{fontenc}
\usepackage[english]{babel}
\usepackage{lmodern}
\usepackage{tikz}

\usetikzlibrary{positioning,calc,arrows}

\usepackage{listings}

\usepackage{amsmath}
\usepackage[labelformat=simple]{subfig}
\usepackage[font={sf},margin=10pt,labelfont=bf]{caption}
\usepackage{booktabs}
\usepackage[colorlinks=false]{hyperref}

\usepackage{fancyhdr}
\setlength{\headheight}{15pt}

\clubpenalty = 10000
\widowpenalty = 10000 \displaywidowpenalty = 10000

\begin{document}

\def\sectionautorefname{Section}
\def\subsectionautorefname{Subsection}

\newcommand\ie{i.e.\ }
\newcommand\eg{e.g.\ }

\definecolor{todo}{rgb}{0.8,0,0}
\newcommand\tod[1]{\texttt{\textcolor{todo}{#1}}}
\newcommand\todo[1]{\par\texttt{\textcolor{todo}{#1}}\par}

  \begin{titlepage}
    \begin{tikzpicture}[remember picture,overlay]
      % Seitenrahmen zeichnen.
      \draw[semithick,rounded corners=0.5cm]
        ($(current page.north west) + ( 1cm,-1cm)$) --
        ($(current page.north east) + (-1cm,-1cm)$) --
        ($(current page.south east) + (-1cm, 1.5cm)$);

      \draw[semithick,rounded corners=0.5cm]
        ($(current page.south east) + (-1cm, 1.5cm)$) --
        ($(current page.south west) + ( 1cm, 1.5cm)$) --
        ($(current page.north west) + ( 1cm,-1cm)$);

      % Logo einbinden.
      \node[anchor=north west] (logo) at ($(current page.north west) + (1.75cm,-1.5cm)$)
      {
        \includegraphics[width=4cm]{KITLogo}
      };

      % Institut / Lehrstuhl.
      \node[anchor=east] at ($(current page.east |- logo.east) + (-1.75cm,0cm)$)
      {
        \begin{minipage}{5.2cm}
          \begin{flushleft}
            \footnotesize{}Institut für Programmstrukturen und Datenorganisation (IPD) \\
            \vspace{6pt}
            Lehrstuhl Prof.\ Dr.-Ing.\ Snelting
          \end{flushleft}
        \end{minipage}
      };

      \node (title) at ($(current page.center |- logo.south) + (0cm, -4cm)$)
      {
        % Korrekter Zeilenabstand etc. durch Minipage.
        \begin{minipage}{12cm}
          \begin{center}
            \huge\textbf{Simple Verification of Rust Programs via Functional Purification}
          \end{center}
        \end{minipage}
      };

      \node[below=1.75cm of title.south]   (prename)  { Masterarbeit von };
      \node[below=0.75cm of prename.south] (name)     { \Large{}\textbf{Sebastian Ullrich} };
      \node[below=1cm    of name.south]    (postname) { an der Fakultät für Informatik };

      \node[below=3cm    of name.south]    (bildchen) { \includegraphics[width=0.9\textwidth]{logo.png}
                                                      };

      \node[below=2cm of bildchen.south] (table)
      {
        \begin{tabular}{ll}
          \textbf{Erstgutachter:}           & Prof.\ Dr.-Ing.\ Gregor Snelting \\[5pt]
          \textbf{Zweitgutachter:}          & ??? \\[5pt]
        \end{tabular}
      };

      \node[below=3.5cm of table.south] (time)
      {
        \begin{tabular}{ll}
        \textbf{Bearbeitungszeit:} & 4. Juli 2013 -- 29. Oktober 2013
        \end{tabular}
      };

      % Fußzeile, unten zentriert.
      \node[anchor=south] (footnote) at ($(current page.center |- current page.south) + (0cm, 0.65cm)$)
      {
        \tiny{}KIT -- Universität des Landes Baden-Württemberg und nationales Forschungszentrum in der Helmholtz-Gemeinschaft
        \hspace{0.5cm}
        \Large{}\textbf{www.kit.edu}
      };
    \end{tikzpicture}
  \end{titlepage}

% sane default for proof documents
\parindent 0pt\parskip 0.5ex

\tikzset{every node/.style={transform shape},auto,block/.style={align=center,rectangle,draw,minimum height=20pt,minimum width=30pt},>=triangle 60}
%\pagenumbering{Roman}
\pagestyle{empty}
\renewcommand{\abstractname}{Einfache Verifikation von Rust-Programmen}
\begin{abstract}
  Imperative Programmiersprachen sind in der modernen Softwareentwicklung
  allgegenwärtig, stellen aber ein Hindernis für formale Softwareverifikation
  dar durch ihre Verwendung von veränderbaren Variablen und Werten. Programme
  in diesen Sprachen können normalerweise nicht direkt auf die unveränderliche
  Welt von Logik und Mathematik zurückgeführt werden, sondern müssen in eine
  explizit modellierte Semantik der jeweiligen Sprache eingebettet werden. Diese
  Indirektion stellt ein Problem für die Benutzung von interaktiven
  Theorembeweisern dar, da sie die Entwicklung von neuen Werkzeugen, Taktiken
  und Logiken für diese ``innere'' Sprache bedingt.

  Die vorliegende Arbeit stellt einen Compiler von der imperativen
  Programmiersprache Rust in die pur funktionale Sprache des Theorembeweisers
  Lean vor, der nicht nur generell das erste Werkzeug zur Verifikation von Rust-Programmen
  darstellt, sondern dies insbesondere auch
  mithilfe der von Lean bereitgestellten Standardwerkzeugen und -logik
  ermöglicht. Diese Transformation ist nur möglich durch spezielle Eigenschaften
  von allen validen Rust-Programmen, die die Veränderbarkeit von Werten auf
  begrenzte Geltungsbereiche einschränken und statisch durch Rusts Typsystem
  garantiert werden. Die Arbeit demonstriert den Einsatz des Compilers anhand
  der Verifikation von Realbeispielen und zeigt die Erweiterbarkeit des Projekts
  über reine Verifikation hinaus am Beispiel von asymptotischer Laufzeitanalyse auf.
\end{abstract}
\renewcommand{\abstractname}{Abstract}

\newpage

\begin{abstract}
  Imperative programming, and aliasing in particular, represents a major
  obstacle in formally reasoning about everyday code. By utilizing restrictions
  the imperative programming language Rust imposes on mutable aliasing, we
  present a scheme for transforming a large part of the Rust language into the
  purely functional language of the Lean theorem prover. We use this scheme to
  verify the correctness of real-world examples of Rust code. We furthermore
  show the extensibility of our transformation by incorporating an analysis of
  asymptotic runtimes.
\end{abstract}
\tableofcontents

\cleardoublepage
\pagestyle{fancy}
\fancyhf{}
\fancyhead[LE,RO]{\thepage}
\fancyhead[RE,LO]{\textit\leftmark}
\pagenumbering{arabic}

\section{Introduction}

Imperative programming languages are ubiquitous in today's software development,
making them prime targets for formal reasoning. Unfortunately, their semantics
differ from those of mathematics and logic -- the languages of formal methods -- in some
significant details, starting with the very concept of ``variables''. The problem
of mutability is only exacerbated for languages that allow references to
\emph{alias}, or point to the same memory location, enabling non-local mutation.

The standard way of verifying programs in such languages with the help of an
interactive theorem prover is to explicitly model the semantics of the language
in the language of the theorem prover, then translate
the program to this representation (a ``deep'' embedding), and finally prove the
correctness of its formalized behavior. This general approach is very flexible
and allows for the verification of meta programs such as program transformations.
The downside of the approach is that the theorem prover's tools and tactics may
not be directly applicable to the embedded language, defeating many amenities of
modern theorem provers.
Alternatively, programs can be ``shallowly'' embedded by directly translating them
into terms in the theorem prover's own language without the use of an explicit inner semantics.
This simplifies many semantic details such as the identification and
substitution of bound variables, but it is harder to accomplish the more the semantics
of the source language differs from that of the theorem prover.
Finally, if a function's specification is given directly in the code using some
sort of annotation, a \emph{verification condition generator} can directly
export the proof obligation as a formula instead of as another program. However,
inserting and maintaining the annotations for a preexisting project may be
troublesome and the specification language is often severely limited because the
verification condition generator has to parse and analyze specifications.

Regardless of the type of embedding,
an explicit heap that references can point into must generally be modeled and
passed around in order to deal with the aliasing problem. References in this model may be as simple as
indices into a uniform heap, but various logics such as separation logic~\cite{reynolds2002separation} have been developed to work on a more abstract representation and to express
aliasing-free sets of references.

Languages with more restricted forms of aliasing do exist, however.
Rust~\cite{matsakis2014rust}, a new, imperative systems programming language,
imposes on mutable references the restriction of never being aliased by any
other reference, mutable or immutable. This restriction eliminates the
possibility of data races and other common bugs created by the presence of
mutable sharing such as iterator invalidation. It furthermore enables a memory-safe
version of manual memory management and more aggressive optimizations.

While the full Rust language also provides raw pointers, which are not bound by
the aliasing restriction, and other ``unsafe'' operations, a
memory model for Rust (informal or formal) has yet to be proposed. We therefore focus on the ``safe''
subset of Rust that has no unsolved semantic details.

We utilize safe Rust's aliasing restriction to design a monadic shallow embedding of a
substantial subset of Rust
into the purely functional language of the Lean~\cite{de2015lean} theorem prover, without the need
for any heap-like indirection. This allows us to
reason about unannotated, real-world Rust code in mostly the same manner one would
reason about native Lean definitions. The monadic approach gives us further
flexibility in modeling additional effects such as function runtime.

We first discuss the simpler cases of the
translation, notably excluding mutable references, in \autoref{sec:trans}. We
show their application by giving a formal verification of Rust's
\rust{[T]::binary_search} method in \autoref{sec:binary_search}.
\autoref{sec:mutref} discusses the translation of most usages of mutable
references, which is used in \autoref{sec:fixedbitset} for a partial verification of the
\texttt{FixedBitSet} data structure. We develop a Lean library for asymptotic analysis in
\autoref{sec:asymptotic} and use it to verify the asymptotic runtime of
\rust{[T]::binary_search}. Lastly, we present some
empirical data about our coverage of the Rust language via its standard library in \autoref{sec:eval}.

\newpage
\bibliographystyle{abbrv}
\bibliography{bib}

\cleardoublepage
\pagestyle{empty}
\section*{Erklärung}

  \vspace{20mm}
  Hiermit erkläre ich, Sebastian Andreas Ullrich, dass ich die vorliegende Masterarbeit selbst\-ständig
verfasst habe und keine anderen als die angegebenen Quellen und Hilfsmittel
benutzt habe, die wörtlich oder inhaltlich übernommenen Stellen als solche kenntlich gemacht und
die Satzung des KIT zur Sicherung guter wissenschaftlicher Praxis beachtet habe.
  \vspace{20mm}
  \begin{tabbing}
  \rule{4cm}{.4pt}\hspace{1cm} \= \rule{7cm}{.4pt} \\
 Ort, Datum \> Unterschrift
  \end{tabbing}

\end{document}

%%% Local Variables:
%%% mode: latex
%%% TeX-master: t
%%% End:
