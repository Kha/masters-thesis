\section{Case Study: Partial Verification of \texttt{FixedBitSet}}
\label{sec:fixedbitset}

Whereas our first case study focused on algorithmic verification, for our second
study we chose the \rust{FixedBitSet} data structure from the \rust{fixedbitset}
crate.\footnote{https://docs.rs/fixedbitset/} It can be thought of as
a more efficient version of a boolean array that stores elements packed at the bit
level. While it is not a complex data structure, verifying it does require
reasoning about the following important parts:

\begin{itemize}
  \item The ubiquitous \rust{Vec} type from the Rust standard library, which
    \rust{FixedBitSet} uses internally, including mutable references into it
  \item Data structure invariants
  \item Bitwise operations
\end{itemize}

\subsection{The Rust Implementation}

\rust{FixedBitSet} uses an internal \rust{Vec} to store up to 32 bits per element.

\begin{minted}{rust}
type Block = u32;

pub struct FixedBitSet {
  data: Vec<Block>,
  /// length in bits
  length: usize,
}
\end{minted}

We will focus on three basic operations: creating (\rust{with_capacity}),
manipulating (\rust{insert}), and querying (\rust{contains}) a
\rust{FixedBitSet}. The Rust implementations are shown in \autoref{lst:fixedbitset}.

\begin{listing}[!bt]
\begin{minted}{rust}
const BITS: usize = 32;
fn div_rem(x: usize, d: usize) -> (usize, usize) {
  (x / d, x % d)
}

impl FixedBitSet {
  pub fn with_capacity(bits: usize) -> Self {
    let (mut blocks, rem) = div_rem(bits, BITS);
    blocks += (rem > 0) as usize;
    FixedBitSet {
      data: vec![0; blocks],
      length: bits,
    }
  }

  pub fn insert(&mut self, bit: usize) {
    assert!(bit < self.length);
    let (block, i) = div_rem(bit, BITS);
    unsafe {
      *self.data.get_unchecked_mut(block) |= 1 << i;
    }
  }

  pub fn contains(&self, bit: usize) -> bool {
    let (block, i) = div_rem(bit, BITS);
    match self.data.get(block) {
      None => false,
      Some(b) => (b & (1 << i)) != 0,
    }
  }
  ...
}
\end{minted}

  \caption{The Rust implementations of the three methods}
  \label{lst:fixedbitset}
\end{listing}

\subsection{Prelude: Axiomatizing \texttt{collections::vec::Vec}}

\rust{Vec} is the standard type for dynamically-sized arrays in Rust. It is
implemented on top of an unsafe abstraction called \rust{RawVec} that handles
allocating, resizing, and deallocating the array memory. \rust{Vec} provides a
safe interface on top of that type by additionally keeping track of ownership of individual items
via a \rust{len} field. Elements after the first \rust{len} items are not
logically part of the \rust{Vec} and must be viewed as unitialized storage.

\begin{minted}{rust}
pub struct Vec<T> {
    buf: RawVec<T>,
    len: usize,
}
\end{minted}

Because \lean{Vec} provides a safe interface, but is itself implemented using
(predominantly) unsafe code, we both can and have to axiomatize it. When axiomatizing data structures, we are free to choose any abstraction as long
as the operations on it preserve their semantics. Just as with arrays and
slices, Lean's basic \lean{list} type is a natural representation for
\rust{Vec}.

\begin{minted}{lean}
structure Vec (T : Type₁) :=
(buf : list T)
\end{minted}

We do lose information about the \rust{RawVec}'s length (also called the
\rust{Vec}'s \emph{capacity}) here, but this information is not exposed by the
\rust{Vec} operations \rust{FixedBitSet} depends on. \autoref{lst:vec} lists the Lean implementations of
the needed \rust{Vec} methods, none of which should be surprising.

\begin{listing}[bt!]
\begin{minted}{lean}
namespace «Vec<T>»
  parameter {T : Type₁}

  definition new : sem (Vec T) :=
  return (Vec.mk [])

  -- note: only a runtime upper bound
  definition push (self : Vec T) (value : T) : sem (unit × Vec T) :=
  sem.incr (list.length (Vec.buf self)) (return (unit.star, Vec.mk (Vec.buf self ++ [value])))

  -- note: `pop` never resizes the `Vec`, so it is always constant-time
  definition pop (self : Vec T) : sem (Vec T × Option T) :=
  match reverse (Vec.buf self) with
  | x :: xs := return (Vec.mk (reverse xs), Option.Some x)
  | []      := return (self, Option.None)
  end

  definition clear (self : Vec T) : sem (Vec T) :=
  sem.incr (list.length (Vec.buf self)) new

  definition len (self : Vec T) : sem usize :=
  return (list.length (Vec.buf self))
end «Vec<T>»
\end{minted}
  
  \caption{Axiomatizations of relevant \rust{Vec} methods}
  \label{lst:vec}
\end{listing}

\rust{Vec} also implements the \rust{Deref} trait, which makes values of type
\rust{&Vec<T>} automatically coerce to \rust{&[T]} and is implicitly being used
in \rust{FixedBitSet::contains}. This is easy enough to implement using our abstraction.

\begin{minted}{lean}
definition «collections.vec.Vec<T> as core.ops.Deref».deref {T : Type₁} (self : Vec T) :
  sem (slice T) :=
return (Vec.buf self)
\end{minted}

There is also a corresponding \rust{DerefMut} trait that makes \rust{&mut
  Vec<T>} coerce to \rust{&mut [T]}. The implementation is slightly more
interesting because it has to return a lens focusing on \lean{Vec.buf}.

\begin{minted}{lean}
definition «collections.vec.Vec<T> as core.ops.DerefMut».deref_mut {T : Type₁} (self : Vec T) :
  sem (lens (Vec T) (slice T) × Vec T) :=
return (⦃lens, get := return ∘ Vec.buf, set := λ old, return ∘ Vec.mk⦄,
        self)
\end{minted}

This trait implementation is being being used by \rust{FixedBitSet::insert} to
access \rust{[T]::get_unchecked_mut}, which in turn returns a mutable reference
to a single slice element.

\begin{minted}{lean}
definition «[T]».get_unchecked_mut {T : Type₁} (self : slice T) (index : usize) :
  sem (lens (slice T) T × slice T) :=
sem.guard (index < length self) (return (lens.index _ index, self))
\end{minted}

This method is interesting in that it actually is unsafe to call in Rust --
instead of an explicit panic, an out-of-bounds access will silently invoke
undefined behavior.

\begin{minted}{rust}
unsafe fn get_unchecked_mut(&mut self, index: usize) -> &mut T {
    &mut *self.as_mut_ptr().offset(index as isize)
}
\end{minted}

There is also a safe, panicking variant called \rust{[T]::get_mut}, which we
first mentioned in \autoref{sec:related} as not being expressible in other
verifiable languages. Because our semantics monad does not currently differentiate between
undefined behavior and panics, both functions become semantically equivalent in
our transformation and we can translate calls to both of them, including the
small bit of unsafe code in \rust{FixedBitSet::insert}.

\subsection{Formal Specification}

There is no useful abstract specification we could give \rust{contains} without
essentially restating its implementation. Instead, we use it to \emph{build} an
abstraction: We translate \rust{FixedBitSet} to a Lean set of indices.

\begin{minted}{lean}
abbreviation sem.returns {A : Type₁} (x : A) := sem.terminates_with (λ a, a = x)

open FixedBitSet

definition to_set (s : FixedBitSet) : set usize :=
{bit | bit < length s ∧ sem.returns bool.tt (contains s bit)}
\end{minted}

The additional constraint \lean{bit < length s} may seem superfluous considering
that \lean{contains} makes sure to always return \rust{false} for indices after
the last \rust{u32} block. However, indices between the length and the capacity
may not necessarily be \rust{false}, as noted in the docstring for a different method:

\begin{minted}{rust}
/// View the bitset as a mutable slice of `u32` blocks. Writing past the
/// bitlength in the last block will cause `contains` to return
/// potentially incorrect results for bits past the bitlength.
pub fn as_mut_slice(&mut self) -> &mut [u32]
\end{minted}

%To make the definition meaningful, we also show that \rust{contains} does
%indeed terminate (for any index).
With \lean{to_set}, we can give \rust{insert} a natural specification using
standard Lean \lean{set} operations.

\begin{minted}{lean}
lemma insert.spec (s : FixedBitSet) (bit : usize) : bit < length s →
  sem.terminates_with
    (λ ret, to_set ret.2 = to_set s ∪ '{bit})
    (insert s bit)
\end{minted}

To prove this lemma, we will also need a data type invariant on \rust{FixedBitSet} relating its two
fields: The \rust{Vec} should always have the minimum length, that is, the
number of bits divided by 32, then rounded up. As with traits, we specify the
invariant as a type class.

\begin{minted}{lean}
structure FixedBitSet' [class] (self : FixedBitSet) : Prop :=
(length_eq : nat.div_ceil (length self) 32 = list.length (Vec.buf (data self)))
\end{minted}

This invariant should be fulfilled by the constructor, \rust{with_capacity}.

\begin{minted}{lean}
lemma with_capacity_inv (bits : usize) [is_usize bits] :
  sem.terminates_with FixedBitSet' (with_capacity bits)
\end{minted}

After adding the hypothesis \lean{[FixedBitSet' s]} to \lean{insert.spec}, the
lemma becomes provable. We also show that the invariant is upheld, \ie that
\lean{FixedBitSet' s'} holds.

\subsection{Proof}

We will focus on the correctness proof of \lean{insert}. With 77 lines, it is
quite shorter (and simpler) than the binary search proof, so we will show some
more details, including some reasoning about bitwise operations.

We again start by unfolding definitions and simplifying the resulting goal. We
also eliminate some bounds checks, introducing \lean{bit_block} for the \rust{u32} block
\lean{bit} is part of, and \lean{l'} and \lean{s'} for the updated list and
\lean{FixedBitSet}, respectively.

\begin{minted}{lean}
...
bit_block : ℕ,
bit_block_eq : list.nth (Vec.buf (FixedBitSet.data s)) (bit / 32) = some bit_block,
l' : list ℕ,
l'_eq : list.update (Vec.buf (FixedBitSet.data s)) (bit / 32) (bit_block ||[32] 2 ^ (bit % 32)) = some l',
s' : FixedBitSet,
s'_eq : s' = FixedBitSet.mk (Vec.mk l') (FixedBitSet.length s)
⊢ FixedBitSet' s' ∧ to_set s' = to_set s ∪ '{bit}
\end{minted}

Here the notation \lean{||[32]} is an abbreviation for \lean{bitor 32}. We show the data type invariant by a helper lemma saying that
\lean{list.length} is invariant under \lean{list.update}. After unfolding
\lean{to_set} and some more simplifications, we are left with a
goal that asserts that some index \lean{bit'} is in the new set iff it is in
the old set or is equal to \lean{bit}.

\begin{minted}{lean}
...
bit' : ℕ
⊢ bit' < FixedBitSet.length s ∧ sem.returns bool.tt (FixedBitSet.contains s' bit') ↔
  bit' < FixedBitSet.length s ∧ sem.returns bool.tt (FixedBitSet.contains s bit') ∨ bit' = bit
\end{minted}

If \lean{bit'} is not a valid index (\lean{bit' ≥ FixedBitSet.length s}), the
goal reduces to \lean{bit' ≠ bit}, which holds because \lean{bit} is assumed to
be valid. If, on the other hand, \lean{bit'} is valid, we will have to reason
about the two \lean{contains} calls. After unfolding them and some more
simplifications, we are left with a bit-level goal talking about the
\rust{u32} block for \lean{bit'} in the old set (\lean{bit'_block}) and in the
new set (\lean{bit'_block'}), respectively.

\begin{minted}{lean}
...
bit'_block bit'_block' : ℕ,
bit'_block_eq : list.nth (Vec.buf (FixedBitSet.data s)) (bit' / 32) = some bit'_block,
bit'_block'_eq : (if bit / 32 = bit' / 32 then some (bit_block ||[32] 2 ^ (bit % 32)) else some bit'_block) = some bit'_block'
⊢ bit'_block' &&[32] 2 ^ (bit' % 32) ≠ 0 ↔
  bit'_block &&[32] 2 ^ (bit' % 32) ≠ 0 ∨ bit' = bit
\end{minted}

We proceed by splitting the goal according to the conditional in
\lean{bit'_block'_eq}. In the case \lean{bit / 32 ≠ bit' / 32}, we obtain
\lean{bit'_block' = bit'_block} and \lean{bit' ≠ bit}, closing the goal. In less
formal words, \lean{bit'} turned out to be in a block entirely unaffected by the
whole insertion.

In the other case, we get \lean{bit'_block = bit_block} and the goal reduces to
a proposition about two bits in the same block.

\begin{minted}{lean}
⊢ (bit_block ||[32] 2 ^ (bit % 32)) &&[32] 2 ^ (bit' % 32) ≠ 0 ↔
  bit_block &&[32] 2 ^ (bit' % 32) ≠ 0 ∨ bit' = bit
\end{minted}

Assuming \lean{bit' = bit}, we see that both sides of the equivalence become universally true.
Otherwise, if \lean{bit' ≠ bit}, but \lean{bit / 32 = bit' / 32} by the previous
assumption, we obtain \lean{bit' % 32 ≠ bit % 32}. A helper lemma proves that this cancels
out the bitwise or and thus reduces both sides to the same term, concluding the proof.