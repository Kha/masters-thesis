\section{Transformation of Mutable References}
\label{sec:mutref}

As the previous section showed, the basic transformation already allows us to reason about mutability in
form of local variables, including inside loops. The next step is to support
non-local or indirect mutability in form of mutable references. We will develop
a restricted but extendable transformation of mutable references in this section
and put it to use in the next section.

\subsection{Lenses as Functional References}

In order to correctly translate mutable references, we will take a more careful
look at their structure in the MIR (\autoref{sec:mir}). Mutable references are
created by the \rust{&mut x} syntax, which in MIR operates on \emph{lvalues}.

\begin{minted}{rust}
pub enum Rvalue<'tcx> {
  /// &x or &mut x
  Ref(&'tcx Region, BorrowKind, Lvalue<'tcx>),
  ...
}
\end{minted}

An lvalue in Rust is either a local or static (global) variable, or inductively some
projection of another lvalue.

\begin{minted}{rust}
pub enum Lvalue<'tcx> {
    Local(Local),
    Static(DefId),
    /// projection out of an lvalue (access a field, deref a pointer, etc)
    Projection(Box<LvalueProjection<'tcx>>),
}
\end{minted}

Because mutable static variables are not allowed in safe Rust, we may assume
that every lvalue is rooted in a local variable. We can describe a mutable
reference as \emph{focusing} on some part of a local variable, which in
functional programming can represented by
\emph{lenses}~\cite{foster2005combinators} (also known as \emph{functional
  references}). For our purposes, a very simple presentation of lenses that
allows us to get and set the focused part is sufficient. We also specialize it
to return our semantics monad.

\begin{minted}{lean}
structure lens (Outer Inner : Type₁) :=
(get : Outer → sem Inner)
(set : Outer → Inner → sem Outer)
\end{minted}

Our lens type describes how some type \lean{Inner} can be extracted from and
replaced inside another type \lean{Outer}. For the correct combinations of those
two types, we can give some general instances such as identity and composition.

\begin{minted}{lean}
definition lens.id {Inner : Type₁} : lens Inner Inner :=
⦃lens, get := return, set := λ o, return⦄

definition lens.comp {A B C : Type₁} (l₂ : lens B C) (l₁ : lens A B) : lens A C :=
⦃lens, get := λ o,
  do o' ← lens.get l₁ o;
  lens.get l₂ o',
 set := λ o i,
  do o' ← lens.get l₁ o;
  do o' ← lens.set l₂ o' i;
  lens.set l₁ o o'⦄

infixr ` ∘ₗ `:60 := lens.comp
\end{minted}

With this, we can translate the \rust{&mut x} operation: We generate a lens per
projection, then compose them together to obtain a value of type \lean{lens A B}
where \lean{B} is the type of \lean{x}, and \lean{A} the type of the root variable of
\lean{x}. For the projection of indexing into an array or slice we can give a generic
definition, but for other projections such as struct fields we will have to
generate them at translation time.

\begin{minted}{lean}
definition lens.index (Inner : Type₁) (index : ℕ) : lens (slice Inner) Inner :=
⦃lens,
  get := λ o, sem.lift_opt (list.nth self o),
  set := λ o i, sem.lift_opt (list.update o index i)⦄
\end{minted}

There is one projection we have to special case: dereferencing an lvalue as in
\rust{*x}. If \rust{x} is an immutable reference, this is just the identity lens
because \rust{&T} and \rust{T} are translated to the same type. If it is a
mutable reference, we compose with its lens to obtain a lens on the ultimate
root variable. This combination of referencing and dereferencing is also known as ``reborrowing''.

\begin{sbs1}
let x: &mut [i32] = &mut a;
let y = &mut (*x)[1];
\end{sbs1}
\begin{sbs2}
let x := lens.id in
let y := lens.index _ 1 ∘ₗ x in
...
\end{sbs2}

There is a final technicality involved with creating mutable references. Because
in Rust a reference is represented merely by an address, index projections are
checked to be in bounds when creating the reference, whereas \lean{lens.index}
will return \lean{mzero} only when its getter or setter is used. Therefore, we
``probe'' lenses eagerly after creation by invoking their getter in order to
make sure we exhibit the same termination behavior as the original code.

\subsection{Pointer Bookkeeping}

In order to actually invoke \lean{lens.get} or \lean{lens.set}, we also need to
pass it the ``outer'' object, \ie the root variable of the original borrow. This
is not a kind of information we can dynamically save alongside the lens in the
mutable reference, but we instead have to statically determine at translation
time. For now, we represent this information as a mapping from variable names to
variable names.

\begin{minted}{rust}
let x: &mut [i32] = &mut a; // {'x' ~> 'a'}
// moving &mut
let x2 = x;                 // {'x2' ~> 'a'}
// reborrowing &mut
let y = &mut (*x2)[1];      // {'x2' ~> 'a', 'y' ~> 'a'}
\end{minted}

While this simple mapping has proved sufficient so far, it does impose
the following limitations:

\begin{itemize}
\item Mutable references can only be stored directly in variables, not nested in
  some structure. This also means that we do not have to worry about how to
  represent mutable references in data types, yet.
\item Whereas a completely static mapping works for linear code, it cannot work
  for variables that are part of a loop state in general. We could lift this
  restriction for the most common special case where the loop changes the lens,
  but not the root variable of a reference.
\end{itemize}

\subsection{Passing Mutable References}

In \autoref{sec:rust}, we introduced references as a more ergonomic (and
efficient) way of passing ownership of a value to some function and getting back the
old or (in the case of mutable references) new value from the function. While we
do not have to worry about ownership in Lean, we can still invert this pattern
for passing mutable references in Lean. For each mutable reference argument, we
read the current value through the lens, pass it to the function, get back the new
value as part of the return value, and write it back through the lens. Inside
the called function, we immediately re-wrap the value in the identity lens.

\begin{sbs1}
fn f(x: &mut T) -> R {...}
...

let x: &mut T = &mut ...a...;
let y = f(x);
\end{sbs1}
\begin{sbs2}
definition f : (xₐ : T) : sem (R × T) :=
let x := lens.id in -- {'x' ~> 'xₐ'}
...

let x = ... in
do tmp ← lens.get x a;
do ret ← f tmp;
match ret with
| (y, tmp) :=
  do a ← lens.set x a tmp;
  ...
end
\end{sbs2}

There is a small caveat with this approach: It does not work if a parameter's type is
declared to be a type parameter, but then instantiated to a mutable reference.

\subsection{Returning Mutable References}

While passing mutable references to functions has a rather simple desugaring,
returning them is a very different beast altogether: The caller has no idea
where the reference is pointing to. For now, we restrict ourselves to the
special case of returning
mutable references that point into the first parameter, which in particular covers all
methods that return references pointing into their \rust{&mut self} parameter.
We statically check this property when translating the callee, and then use that
knowledge in the caller to compose the returned lens with the lens of the first argument. Note
that we still have to return the new pointee for the first argument, as by the
previous subsection.

\begin{sbs1}
fn f(x: &mut T) -> &mut R {...}
...

let x: &mut T = &mut ...a...;
let y = f(x);
\end{sbs1}
\begin{sbs2}
definition f : (xₐ : T) : sem (lens T R × T) := ...

let x = ... in
do tmp ← lens.get x a;
do ret ← f tmp;
match ret with
| (y, tmp) :=
  let y := y ∘ₗ x in
  do a ← lens.set x a tmp;
  ...
end
\end{sbs2}

% \subsection{Discussion}
% 
% The astute reader may notice that our lens representation of mutable references
% does not actually make use of their uniqueness property; while we do utilize the
% absence of aliasing for representing the more common immutable references
% (\autoref{sec:refs}), two lenses focused one the same object could in theory
% coexist because they do not hold any state on their own.
% 
% We could indeed make use of the aliasing property by saving the state in the
% lens itself and writing it back only at the lens' end of lifetime.
% 
% \begin{minted}{lean}
% structure lens (Outer Inner : Type₁) :=
% (value : Inner)
% (set : Outer → Inner → sem Outer)
% \end{minted}
% 
% While this arguably simpler representation would save us a monadic bind when
% reading from a mutable reference, there are one technical reason and two arguments
% of future extensibility that made us go with the former representation. 