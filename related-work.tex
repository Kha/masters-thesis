\section{Related Work}

While this thesis presents the first verification tool for Rust programs, tools
for many other imperative languages have been developed before.

The Why3 project~\cite{bobot2011why3} is notable for its generality. It provides
an imperative ML-like language \emph{WhyML} together with a verification
condition generator that can interface with a multitude of both automatic and
interactive theorem provers. While WhyML supports advanced language features such
as type polymorphism and exceptions, it does not support higher-order functions,
which are ubiquitous in Rust code.
%Specification annotations of WhyML programs are written in the logic language \emph{Why3} and thus cannot 
WhyML provides a reference type \texttt{ref} that can point to a fresh cell on
the heap and is statically checked not to alias with other \texttt{ref}
instances, but cannot point into some existing datum like Rust references can.
For example, the first of the following two WhyML functions fails to type check
because the array elements are not known to be alias-free, while the second one
will return a reference to a \emph{copy} of \verb!a[i]!.

\begin{minted}{sml}
let get_mut (a : array (ref int)) (i : int) : ref int = a[i]
let get_mut (a : array int) (i : int) : ref int = ref a[i]
\end{minted}

In contrast, Rust can provide a perfectly safe function with this functionality.

\begin{minted}{rust}
fn slice::get_mut<T>(slice: &mut [T], index: usize) -> Option<&mut T>
\end{minted}

WhyML is also being used as an intermediate language for the verification of
programs in Ada~\cite{guitton2011hi}, C~\cite{cuoq2012frama} and Java~\cite{filliatre2007krakatoa}.
For the latter two languages, aliasing is reintroduced by way of an explicit heap.

The remarkable SeL4 project~\cite{klein2009sel4} delivers a full formal verification of an operating
system microkernel by way of multiple levels of program verification and
refinement steps. The C code that produces the final kernel binary is verified
by embedding it into the theorem prover
Isabelle/HOL~\cite{nipkow2002isabelle}, using a deep embedding for statements
and a shallow one for expressions. The C memory model used allows type-unsafe
operations by use of a byte-size heap, but as with Why3, higher-order functions are
not supported. The AutoCorres~\cite{greenaway2012bridging, greenaway2014don}
tool then transforms this representation into a shallow monadic embedding,
dealing with the `uninteresting complexities of C'~\cite{greenaway2014don} on the
way. The result is an abstracted representation that is quite similar to ours
(and in fact inspired it in part, as we shall note below), but doesn't go the
last mile of completely eliminating the heap where possible. Without explicit
no-alias annotations, the semantics of C would allow that in far fewer places than those
of Rust in any case.

It should be noted that our work, like most verification projects based on
either embedding or code extraction, relies on both
an unverified compiler and an unverified embedding tool, effectively making both
part of the trusted computing base. SeL4 is a notable exception in this,
providing (at lower optimization levels) a direct equivalence proof~\cite{sewell2013translation} between the
produced kernel binary and the verified embedded code, thus completely removing
the original C code from the trusted computing base.

While not an imperative language, the purely functional, total Cogent language~\cite{o2016refinement}
uses linear types in the style of Wadler~\cite{wadler1990linear} to avoid
automatic memory management, much like Rust. Building
on AutoCorres, the language is designed both to be easily verifiable and to
compile down to efficient C code. As we shall see in \autoref{sec:rust}, the
biggest differences between Wadler-style purely functional linear languages and Rust are the
existence of mutable references as well as sophisticated interprocedural
reference tracking in the latter. For example, the aforementioned \rust{get_mut}
function can only be expressed as a higher-order function in Cogent, even in the immutable case.